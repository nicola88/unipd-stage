\documentclass[10pt,a4paper,hidelinks]{scrartcl} % KOMA-Script article scrartcl
\usepackage[italian]{babel}
\usepackage[utf8x]{inputenc}
\usepackage[nochapters]{../classicthesis} % nochapters
\setcounter{tocdepth}{1}

\begin{document}
    \title{\rmfamily\normalfont\spacedallcaps{Piano di lavoro}}
    \author{\spacedlowsmallcaps{Nicola Moretto (matr. 578258)}}
    \date{\today}
    
    \maketitle
    
    \begin{abstract}
        \noindent Piano di lavoro settimanale per stage esterno presso \textsf{FondaMente}.
    \end{abstract}
    
	\tableofcontents

    \section{Contesto lavorativo}
	Analisi, progettazione e sviluppo di una piattaforma web tematica per la condivisione di contenuti informativi creati dagli utenti e la vendita diretta di prodotti da parte di aziende medio-piccole o indipententi.
	
	La pianificazione e l'esecuzione delle attività di progetto è gestita mediante software \textit{Spark} e \textit{Altova}.

	Lo studente incontrerà il proponente al termine di ciascuna settimana di lavoro per valutare i progressi e manterrà contatti quotidiani con gli altri membri del team di progetto.

	\section{Obiettivi}
	\begin{enumerate}
		\item Estendere l'attuale sistema di classificazione della piattaforma al fine di integrare criteri addizionali che facilitino la catalogazione e il recupero dei contenuti informativi.
		\item Progettare un'interfaccia grafica dinamica per la navigazione dei contenuti, che presenti:
		\begin{itemize}
			\item le informazioni essenziali associate a ciascun contenuto;
			\item le relazioni esistenti tra i contenuti;
			\item i criteri e i parametri del sistema di classificazione;
			\item \ldots
		\end{itemize}
	\end{enumerate}

	\newpage
	\section{Attivit\`a}
	
	\subsection{Analisi delle classi di contenuti informativi della piattaforma}
	\label{sec:stage:wp:analisi-contenuti}
	A partire dall'insieme di classi di contenuti informativi presenti nella piattaforma, verrà svolta un'analisi dettagliata per evidenziarne analogie e differenze, con particolare attenzione agli aspetti di interesse e rilevanza per il sistema di classificazione:
	\begin{itemize}
		\item scenario d'uso;
		\item proprietà\footnote{Autore, data di creazione, titolo, contenuto, emozione, intenzioni, giudizio, \ldots};
		\item interrelazioni;
		\item \ldots
	\end{itemize}

	\subsection{Analisi delle specifiche del sistema di classificazione}
	\label{sec:stage:wp:ar}
	Verranno individuati i requisiti del sistema di classificazione dei contenuti in accordo ai risultati dell'analisi svolta nella fase \ref{sec:stage:wp:analisi-contenuti} e a valutazioni preliminari connesse alla facilità di reperimento delle informazioni e di navigazione dei contenuti.

	\subsection{Analisi comparativa dei principali sistemi di classificazione della conoscenza}
	\label{sec:stage:wp:ac}
	Verrà svolta un'analisi comparativa di alcuni sistemi di classificazione esistenti per valutarne punti di forza e debolezza rispetto ai requisiti emersi nella fase \ref{sec:stage:wp:ar}:
	\begin{itemize}
		\item dizionario (\emph{tag});
		\item tassonomia (\emph{argomenti/categorie});
		\item \ldots
	\end{itemize}
	La selezione definitiva dei sistemi di classificazione da prendere in esame verrà concordata con il referente sulla base dei requisiti sopra citati.

	\subsection{Progettazione del sistema di classificazione}
	\label{sec:stage:wp:design}
	Verrà svolta la fase iniziale di progettazione del sistema di classificazione della conoscenza, con particolare evidenza per i punti di forza dei sistemi analizzati nella fase \ref{sec:stage:wp:ac} e per le soluzioni adottate per superarne limiti o vincoli.
	% Argomenti + grafo orientato di parole chiave definite dall'utente o ricavate dal testo

	Dovranno essere inoltre tenute in debita considerazione le problematiche legate alle prestazioni del sistema in scenari caratterizzati da un numero consistente di contenuti da gestire, elaborare e visualizzare.
	
	\subsection{Implementazione del sistema di classificazione nel modello relazionale}
	\label{sec:stage:wp:build}
	Verranno integrate - nel modello relazionale esistente della piattaforma - le informazioni aggiuntive necessarie ad implementare il sistema di classificazione, così come definito al termine della fase \ref{sec:stage:wp:design}.

	Le modifiche introdotte riguarderanno innanzitutto l'adeguamento delle entità relative ai tipi di contenuti e al sistema di classificazione per gestire i nuovi criteri di classificazione.

	\subsection{Analisi dei requisiti dell'interfaccia grafica}
	\label{sec:stage:wp:ui:ar}
	Verrà svolta un'analisi dei requisiti per individuare i casi d'uso e i requisiti dell'interfaccia grafica, a partire dalle specifiche dei contenuti e del sistema di classificazione (fase \ref{sec:stage:wp:ar}), in particolare rispetto alla visualizzazione, la navigazione e il filtraggio dei contenuti.

	\subsection{Progettazione dell'interfaccia grafica: visualizzazione dei contenuti}
	\label{sec:stage:wp:ui:design:view}
	Verrà svolta la fase iniziale di progettazione dell'interfaccia grafica, che prevede il soddisfacimento dei requisiti inerenti la visualizzazione dei contenuti della piattaforma (classe, proprietà, interrelazioni, \ldots).
	
	La valutazione degli aspetti grafici e visivi è oggetto di studio da parte di altre figure professionali (psicologi e sociologi) coninvolti nel progetto.
	%(colore, forma, \ldots)

	\subsection{Progettazione dell'interfaccia grafica: filtraggio dei contenuti}
	\label{sec:stage:wp:ui:design:filter}
	Verrà portata avanti la fase di progettazione dell'interfaccia grafica, con la definizione e l'integrazione delle opzioni e degli strumenti per filtrare i contenuti visualizzati, secondo i criteri di classificazione della piattaforma e le proprietà fondamentali dei contenuti medesimi.

	\subsection{Progettazione dell'interfaccia grafica: navigazione dei contenuti}
	\label{sec:stage:wp:ui:design:nav}
	Verrà portata a termine la fase di progettazione dell'interfaccia grafica, al fine di offrire un meccanismo di navigazione di contenuti affini in base ai criteri di classificazione definiti nella fase \ref{sec:stage:wp:design}.

	\newpage
	\section{Calendario}
	
	\subsection*{I settimana (10/09/2012 - 14/09/2012)}
	\begin{enumerate}
	\item \nameref{sec:stage:wp:analisi-contenuti}
	\item \nameref{sec:stage:wp:ar}
	\end{enumerate}
	\subsection*{II settimana (17/09/2012 - 21/09/2012)}
	\begin{enumerate}
	\item \nameref{sec:stage:wp:ac}
	\end{enumerate}
	\subsection*{III settimana (24/09/2012 - 28/09/2012)}
	\begin{enumerate}
	\item \nameref{sec:stage:wp:design}
	\end{enumerate}
	\subsection*{IV settimana (1/10/2012 - 5/10/2012)}
	\begin{enumerate}
	\item \nameref{sec:stage:wp:build}
	\end{enumerate}
	\subsection*{V settimana (8/10/2012 - 12/10/2012)}
	\begin{enumerate}
	\item \nameref{sec:stage:wp:ui:ar}
	\end{enumerate}	
	\subsection*{VI settimana (15/10/2012 - 19/10/2012)}
	\begin{enumerate}
	\item \nameref{sec:stage:wp:ui:design:view}
	\end{enumerate}
	\subsection*{VII settimana (22/10/2012 - 26/10/2012)}
	\begin{enumerate}
	\item \nameref{sec:stage:wp:ui:design:filter}
	\end{enumerate}
	\subsection*{VIII settimana (29/10/2012 - 2/11/2012)}
	\begin{enumerate}
	\item \nameref{sec:stage:wp:ui:design:nav}
	\end{enumerate}

	\begin{table}[ht]
	\centering
	\begin{tabular}{|c|c|}
	\hline
	\textsc{Settimana} & \textsc{Ore di lavoro} \\ \hline
	I & 40 \\ \hline
	II & 40 \\ \hline
	III & 40 \\ \hline
	IV & 40 \\ \hline
	V & 40 \\ \hline
	VI & 40 \\ \hline
	VII & 40 \\ \hline
	VIII & 40 \\ \hline
	\textsc{Totale} & 320 \\ \hline
	\end{tabular}
	\caption{Ore lavorative}
	\label{tab:stage:wp:workload}
	\end{table}


	\newpage
	\section{Prodotti}
	Di seguito si riporta per ciascun prodotto previsto il \textsc{tipo} (documentazione o software) e le \textsc{fasi} al termine delle quali è previsto il rilascio di una versione, indicata con la notazione \textsc{x.y} (ove \textsc{x.0} rappresenta una versione approvata dal proponente).

	\subsection{Sistema di classificazione}
	\label{sec:stage:prod:sc}
	La prima versione (1.0) del documento descrive i requisiti del sistema di classificazione, frutto di analisi preliminari sui contenuti e le relative relazioni.

	Nella versione successiva (2.0) tali informazioni vengono integrate e ampliate con i risultati dell'analisi di alcuni sistemi di classificazione esistenti, con particolare riguardo ai punti di forza e ai limiti rispetto ai suddetti requisiti, e vengono quindi illustrate e motivate in dettaglio le scelte progettuali compiute, tenendo in considerazione gli eventuali vincoli connessi alla complessità di elaborazione delle informazioni di classificazione nel contesto reale della piattforma.
	\begin{description}
	\item[Fase (versione):] \ref{sec:stage:wp:ar} (1.0) - \ref{sec:stage:wp:design} (2.0)
	\item[Tipo:] documentazione
	\end{description}

	\subsection{Modello relazionale}
	Il documento presenta le modifiche richieste al modello relazionale esistente della piattaforma per integrare i nuovi criteri, discutendone vantaggi e svantaggi.
	\label{sec:stage:prod:mr}
	\begin{description}
	\item[Fase (versione):] \ref{sec:stage:wp:build} (1.0)
	\item[Tipo:] documentazione
	\end{description}

	\subsection{Interfaccia grafica - Analisi dei requisiti}
	Il documento descrive i requisiti e i casi d'uso dell'interfaccia grafica per la visualizzazione, la navigazione e il filtraggio dei contenuti alla luce dei nuovi criteri di classificazione integrati nella piattaforma.
	\label{sec:stage:prod:ui:ar}
	\begin{description}
	\item[Fase (versione):] \ref{sec:stage:wp:ui:ar} (1.0)
	\item[Tipo:] documentazione
	\end{description}

	\subsection{Interfaccia grafica - Schema di progettazione}
	Il documento fornisce le linee guida dettagliate per l'implementazione dell'interfaccia grafica, tra cui:
	\begin{itemize}
	\item la rappresentazione dei \textit{contenuti} (tipo, proprietà, relazioni, \ldots) e \textit{criteri di classificazione} (valori, relazioni, \ldots);
	\item la definzione dei gradi di libertà e delle modalità d'interazione per la navigazione dei contenuti affini.
	\end{itemize}
	Le informazioni riportate prescinderanno da qualsiasi assunzione riguardo la tecnologia utilizzata, fornendo una descrizione astratta dei principi e delle specifiche di funzionamento.
	\label{sec:stage:prod:ui:design}
	\begin{description}
	\item[Fase (versione):] \ref{sec:stage:wp:ui:design:view} (1.0) - \ref{sec:stage:wp:ui:design:filter} (2.0) - \ref{sec:stage:wp:ui:design:nav} (3.0)
	\item[Tipo:] documentazione
	\end{description}

	%\section{Conoscenze acquisite}

\end{document}
