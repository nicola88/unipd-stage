\documentclass[10pt,a4paper,hidelinks]{scrartcl} % KOMA-Script article scrartcl
\usepackage[italian]{babel}
\usepackage[utf8x]{inputenc}
\usepackage[nochapters]{../classicthesis} % nochapters
\setcounter{tocdepth}{1}

\begin{document}
    \title{\rmfamily\normalfont\spacedallcaps{Piano di lavoro}}
    \author{\spacedlowsmallcaps{Nicola Moretto (matr. 578258)}}
    \date{\today}
    
    \maketitle
    
    \begin{abstract}
        \noindent Piano di lavoro settimanale per stage esterno presso \textsf{FondaMente}.
    \end{abstract}
    
	\tableofcontents

    \section{Contesto lavorativo}
	Analisi, progettazione e sviluppo di una piattaforma web tematica per la condivisione di contenuti informativi creati dagli utenti e la vendita diretta di prodotti da parte di aziende medio-piccole o indipententi.

	\section{Obiettivi}
	\begin{enumerate}
		\item Estendere l'attuale sistema di classificazione della piattaforma al fine di integrare criteri addizionali che facilitino la catalogazione e il recupero dei contenuti informativi.
		\item Progettare un'interfaccia grafica dinamica per la consultazione dei risultati di una ricerca, che presenti:
		\begin{itemize}
			\item le informazioni essenziali associate a ciascun contenuto;
			\item le relazioni esistenti tra i contenuti;
			\item i criteri e i parametri del sistema di classificazione;
			\item \ldots
		\end{itemize}
	\end{enumerate}

	\section{Attivit\`a}
	
	\subsection{Analisi delle classi di contenuti informativi della piattaforma}
	\label{sec:stage:wp:analisi-contenuti}
	A partire dall'insieme di classi di contenuti informativi presenti nella piattaforma, verrà svolta un'analisi dettagliata per evidenziarne analogie e differenze, con particolare attenzione agli aspetti di interesse e rilevanza per il sistema di classificazione:
	\begin{itemize}
		\item scenario d'uso;
		\item proprietà\footnote{Autore, data di creazione, titolo, contenuto, emozione, intenzioni, giudizio, \ldots};
		\item interrelazioni;
		\item \ldots
	\end{itemize}

	\subsection{Analisi delle specifiche del sistema di classificazione}
	\label{sec:stage:wp:ar}
	Verranno individuati i requisiti del sistema di classificazione dei contenuti in accordo ai risultati dell'analisi svolta nella fase \ref{sec:stage:wp:analisi-contenuti} e a valutazioni preliminari connesse alla facilità di reperimento delle informazioni e di navigazione dei contenuti.

	\subsection{Analisi comparativa dei principali sistemi di classificazione della conoscenza}
	\label{sec:stage:wp:ac}
	Verrà svolta un'analisi comparativa di alcuni sistemi di classificazione esistenti per valutarne punti di forza e debolezza rispetto ai requisiti emersi nella fase \ref{sec:stage:wp:ar}:
	\begin{itemize}
		\item dizionario (\emph{tag});
		\item tassonomia (\emph{argomenti/categorie});
		\item \ldots
	\end{itemize}
	La selezione definitiva dei sistemi di classificazione da prendere in esame verrà stilata di comune accordo con il referente sulla base dei requisiti sopra citati.

	\subsection{Progettazione del sistema di classificazione}
	\label{sec:stage:wp:design}
	Verrà svolta la fase iniziale di progettazione del sistema di classificazione della conoscenza, con particolare evidenza per i punti di forza dei sistemi analizzati nella fase \ref{sec:stage:wp:ac} e per le soluzioni adottate per superarne limiti o vincoli.
	% Argomenti + grafo orientato di parole chiave definite dall'utente o ricavate dal testo

	Dovranno essere inoltre tenute in debita considerazione le problematiche legate a usabilità e prestazioni del sistema in scenari caratterizzati da un numero consistente di contenuti da gestire, elaborare e visualizzare.
	
	\subsection{Implementazione del sistema di classificazione nel modello relazionale}
	\label{sec:stage:wp:build}
	Verranno integrate - nel modello relazionale esistente della piattaforma - le informazioni aggiuntive necessarie ad implementare il sistema di classificazione, così come definito al termine della fase \ref{sec:stage:wp:design}.

	Le modifiche introdotte riguarderanno innanzitutto l'adeguamento delle entità relative ai tipi di contenuti e al sistema di classificazione per gestire i nuovi criteri di classificazione.

	\subsection{Analisi dei requisiti dell'interfaccia grafica}
	\label{sec:stage:wp:ui:ar}
	Verrà stilata la lista dei requisiti dell'interfaccia grafica a partire dalle specifiche del sistema di classificazione (fase \ref{sec:stage:wp:ar}), in particolare rispetto alla visualizzazione dei contenuti e al raffinamento dei criteri di ricerca.

	\subsection{Progettazione dell'interfaccia grafica: visualizzazione dei risultati di ricerca}
	\label{sec:stage:wp:ui:design}
	Verrà svolta la fase iniziale di progettazione dell'interfaccia grafica puntando al soddisfacimento dei requisiti inerenti la visualizzazione delle informazioni rilevanti dei contenuti della piattaforma (classe, proprietà, interrelazioni, \ldots) e il sistema di classificazione (criteri, parametri, \ldots).
	%(colore, forma, \ldots)

	\subsection{Progettazione dell'interfaccia grafica: raffinamento dei criteri di ricerca}
	\label{sec:stage:wp:ui:advanced-design}
	Verrà portata a termine la progettazione dell'interfaccia grafica con il soddisfacimento dei requisiti rimanenti, correlati essenzialmente al raffinamento e alla manipolazione dei risultati della ricerca.

	In questa fase dovranno essere individuati i criteri e i parametri su cui l'utente potrà intervenire per ampliare o restringere l'ambito della ricerca e dovranno essere definite le modalità e i gradi di libertà con cui potrà farlo.

	\section{Calendario}
	
	\subsection*{I settimana (10/09/2012 - 14/09/2012)}
	\begin{enumerate}
	\item \nameref{sec:stage:wp:analisi-contenuti}
	\item \nameref{sec:stage:wp:ar}
	\end{enumerate}
	\subsection*{II settimana (17/09/2012 - 21/09/2012)}
	\begin{enumerate}
	\item \nameref{sec:stage:wp:ac}
	\end{enumerate}
	\subsection*{III settimana (24/09/2012 - 28/09/2012)}
	\begin{enumerate}
	\item \nameref{sec:stage:wp:design}
	\end{enumerate}
	\subsection*{IV settimana (1/10/2012 - 5/09/2012)}
	\begin{enumerate}
	\item \nameref{sec:stage:wp:build}
	\end{enumerate}
	\subsection*{V settimana (8/10/2012 - 12/10/2012)}
	\begin{enumerate}
	\item \nameref{sec:stage:wp:ui:ar}
	\end{enumerate}	
	\subsection*{VI settimana (15/10/2012 - 19/10/2012)}
	\begin{enumerate}
	\item \nameref{sec:stage:wp:ui:design}
	\end{enumerate}
	\subsection*{VII settimana (22/10/2012 - 26/10/2012)}
	\begin{enumerate}
	\item \nameref{sec:stage:wp:ui:design}
	\item \nameref{sec:stage:wp:ui:advanced-design}
	\end{enumerate}
	\subsection*{VIII settimana (29/10/2012 - 2/1/2012)}
	\begin{enumerate}
	\item \nameref{sec:stage:wp:ui:advanced-design}
	\end{enumerate}

	%\section{Conoscenze acquisite}
	%\ldots %Competenze di analisi di sistemi di classificazione

\end{document}
