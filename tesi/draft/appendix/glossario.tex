\chapter{Glossario}
\label{ch:appendice:glossario}

\section*{B}
\paragraph{Bitbucket - \url{https://bitbucket.org/}} \hfill \\
Piattaforma web per la gestione delle attività di progetto con supporto a strumenti di controllo di versione distribuito.

\section*{C}
\paragraph{CamelCase} \hfill \\
Convenzione per la scrittura di espressioni composte unendo le parole tra loro e mantenendo ciascuna iniziale in maiuscolo.

\paragraph{Chilometri zero} \hfill \\
Filosofia di consumo basata sull'acquisto di beni agroalimentari direttamente dal produttore, evitando la filiera, tutelando l'ambiente e valorizzando la produzione locale del territorio.

\section*{G}
\paragraph{gedit - \url{http://projects.gnome.org/gedit/}} \hfill \\
Editor di testo ufficiale dell'ambiente desktop GNOME.
\paragraph{Gruppi di acquisto} \hfill \\
Insieme (stabile o provvisorio) di consumatori che acquista mediante ordine collettivo un consistente numero di beni direttamente dal produttore, spesso per conseguire prezzi vantaggiosi o ammortizzare eventuali spese accessorie.

\section*{L}
\paragraph{LaTeXila - \url{http://projects.gnome.org/latexila/}} \hfill \\
Editor LaTex integrato per l'ambiente desktop GNOME.
\paragraph{LibreOffice Calc - \url{http://www.libreoffice.org/}} \hfill \\
Applicazione per fogli di calcolo della suite di produttività \textit{LibreOffice}.

\section*{M}
\paragraph{Mercurial - \url{http://mercurial.selenic.com/}} \hfill \\
Strumento multi piattaforma, gratuito ed open source per il controllo di versione distribuito.
\paragraph{MySQL Workbench - \url{http://www.mysql.it/products/workbench/}} \hfill \\
Applicazione multi piattaforma, gratuita ed open source per la progettazione, lo sviluppo e l'amministrazione di database MySQL. 

\section*{P}
\paragraph{PDF (Portable Document Format)} \hfill \\
Formato di file per la rappresentazione di documenti in maniera indipendente dalla piattaforma hardware e software.
\paragraph{Pencil - \url{http://pencil.evolus.vn/}} \hfill \\
Applicazione multi piattaforma, gratuita ed open source per la realizzazione di prototipi di interfacce grafiche.
\paragraph{ProjectLibre - \url{http://sourceforge.net/projects/projectlibre/}} \hfill \\
Applicazione multi piattaforma, gratuita ed open source per il \textit{project management}, che consente di realizzare diagrammi di Gantt e di PERT, gestire le risorse allocate e le attività pianificate, \ldots\ .

\section*{R}
\paragraph{Rete sociale} \hfill \\
Insieme di persone, aventi interessi in comune e inclini a collaborare e condividere idee o informazioni, e di relazioni di tipo esperienziale definite tra tali soggetti.

\section*{U}
\paragraph{Ubuntu - \url{http://www.ubuntu.com/}} \hfill \\
Distribuzione Linux gratuita derivata da Debian.
\paragraph{UML (Unified Modelling Laanguage) - \url{http://www.uml.org/}} \hfill \\
Standard internazionale per un linguaggio di modellazione, che definisce un insieme di notazioni grafiche per la rappresentazione visiva di sistemi.
\paragraph{UMLet - \url{http://www.umlet.com/}} \hfill \\
Applicazione multi piattaforma, gratuita ed open source per la realizzazione di diagrammi UML.
\paragraph{UTF-8 (Unicode Transformation Format-8) - \url{http://www.unicode.org/standard/}} \hfill \\
Codifica dei caratteri Unicode a 8 bit. Si distingue dalla maggior parte delle altre codifiche per la capacità di rappresentare un insieme più ampio di caratteri, non limitato ad una specifica area geografica o ad un singolo alfabeto.
