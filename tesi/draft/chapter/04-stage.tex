\chapter{Stage}
\label{ch:tesi:stage}

\section{Piano di stage}
\label{sec:tesi:stage:piano}

\subsection{Obiettivi e requisiti}
\label{sec:tesi:stage:piano:obiettivi}
L'attività di stage si colloca nell'ambito del progetto presentato nel capitolo \ref{ch:tesi:progetto} perseguendo due obiettivi distinti ma correlati e focalizzandosi sulla componente \textit{social} della piattaforma.

Secondo alcune stime e indagini interne al team di progetto, la piattaforma presenta un potenziale bacino di utenti piuttosto ampio: ci si aspetta dunque che l'attività e le dimensioni (in termini, ad esempio, di numero di contenuti pubblicati) possano attestarsi su livelli tali da imporre valutazioni di carattere tecnico, atte a garantire che il sistema sia in grado di gestire in modo accettabile il traffico generato.

Sebbene la caratterizzazione quantitativa e qualitativa del problema e degli eventuali limiti imposti sia tuttora oggetto di analisi, ci sono evidenti implicazioni e ricadute per quanto concerne l'attività di stage, che deve cercare, pur in modo preliminare ed empirico, di esaminare le potenziali criticità, l'incidenza e l'impatto delle scelte sulle prestazioni e sull'efficienza del sistema.

\subsubsection{Criteri di classificazione}
\label{sec:tesi:stage:piano:obiettivi:criteri}
Il primo obiettivo consiste nell'estendere l'attuale sistema di classificazione (v. sezione \ref{sec:progetto:classificazione}) integrandovi un criterio aggiuntivo per la catalogazione del patrimonio di conoscenza della piattaforma e la costruzione di un'enciclopedia del sapere a partire dai contenuti pubblicati dagli utenti al fine di rendere il reperimento e la consultazione delle informazioni desiderate il più efficienti ed agevoli possibile.

L'ideazione e concezione di tale criterio deve tener conto della natura tematica della piattaforma, riuscendo a conciliare due esigenze distinte:
\begin{itemize}
\item dev'essere sufficientemente astratto e flessibile per adattarsi alla molteplicità di varianti tematiche in cui la piattaforma può essere declinata;
\item dev'essere ottimizzato per avvantaggiarsi delle peculiarità di una piattaforma tematica, ad esempio la maggior correlazione degli argomenti trattati.
\end{itemize}

La soluzione individuata deve prescindere da assunzioni legate alla tecnologia utilizzata. Inoltre, alla luce di possibili evoluzioni nello sviluppo della piattaforma, si desidera che la classificazione di un contenuto informativo (assegnazione di metadati, individuazione di correlazioni, \ldots) possa essere - in futuro - demandata a componenti software integrate nella piattaforma.

\subsubsection{Interfaccia grafica}
\label{sec:tesi:stage:piano:obiettivi:interfaccia}
Il secondo obiettivo consiste nel progettare un'interfaccia grafica per la consultazione dei contenuti informativi, che sfrutti il criterio di classificazione aggiuntivo per facilitare la ricerca ed il reperimento delle informazioni di interesse per l'utente all'interno del patrimonio enciclopedico della piattaforma. La sfida principale consiste nel progettare un'interfaccia altamente scalabile, ossia in grado di visualizzare in maniera chiara e ordinata un elevato numero di contenuti in svariate classi di dispositivi (\textit{smartphone},\textit{tablet},\textit{notebook},\ldots).

Il primo passo consiste nell'individuare le informazioni essenziali per una rapida e precisa identificazione di un contenuto (titolo, autore, data, \ldots) e valutare quindi la notazione (grafica o testuale) più adatta per esprimerle, rendendole accessibili al maggior numero possibile di utenti; le informazioni aggiuntive devono essere comunque accessibili, ma solo su esplicita richiesta dell'utente. In questo ambito si inseriscono una serie di analisi e valutazioni di carattere sociologico per individuare le soluzioni più adatte per comunicare tali informazioni in modo da renderne la comprensione chiara e intuitiva agli utenti della piattaforma.

Il passo successivo consiste nel rendere l'interfaccia facilmente navigabile, mostrando in maniera ordinata e intuitiva i contenuti e le reciproche relazioni, che formano concettualmente una struttura a grafo orientato (v. sezione \ref{sec:progetto:contenuti}). Occorre perciò individuare opportuni criteri di raggruppamento, ordinamento e collocamento dei contenuti visualizzati per favorirne la ricerca o la consultazione e fissare eventuali limiti dinamici per evitare un eccessivo affollamento dell'interfaccia e garantire un livello adeguato di leggibilità.%\footnote{I limiti dinamici possono essere vincolati a svariati parametri, relativi al sistema software o al dispositivo utilizzato (potenza di calcolo, risoluzione o diagonale dello schermo, \ldots).}

Il terzo ed ultimo passo consiste nell'aggiungere la possibilità per l'utente di filtrare i contenuti mostrati in accordo a proprietà (argomento, autore, data di pubblicazione, tipo) o metadati associati (attinenza, emozioni, giudizi, intenzioni).

Per individuare i requisiti essenziali si inizia prendendo in considerazione alcuni casi d'uso classici:
\begin{enumerate}
\item l'utente naviga liberamente tra i contenuti (più recenti, più letti, più discussi, \ldots);
\item l'utente consulta la discussione generata da un contenuto;
\item l'utente cerca le informazioni riguardanti un certo tema (contenuti affini, \ldots);
\item l'utente esplora le relazioni esistenti tra i temi trattati e discussi; %(tema generico, \ldots).
\end{enumerate}

\subsection{Pianificazione}
L'attività di stage viene suddivisa in due fasi distinte per semplificarne la pianificazione:
\begin{enumerate}
\item l'estensione del sistema di classificazione;
\item l'analisi e la progettazione dell'interfaccia grafica.
\end{enumerate}

Per ciascuna fase sono fissati gli obiettivi generali, sono individuate e organizzate su base settimanale le attività da svolgere, cercando di garantire un carico di lavoro equilibrato, e sono indicati i prodotti attesi.

La durata complessiva dello stage si attesta su 8 settimane a tempo pieno, corrispondenti a 320 ore di lavoro.

\begin{table}[ht]
\centering
\begin{tabular}{|p{10cm}|c|}
\hline
\textsc{Attività} & \textsc{Ore di lavoro} \\ \hline
\multicolumn{2}{|c|}{\textit{Fase 1: estensione del sistema di classificazione}} \\ \hline 
Analisi delle specifiche del sistema di classificazione & 40 \\ \hline
Analisi comparativa dei principali sistemi di classificazione della conoscenza & 40 \\ \hline
Progettazione del sistema di classificazione & 40 \\ \hline
Implementazione del sistema di classificazione nel modello relazionale & 40 \\ \hline
\multicolumn{2}{|c|}{\textit{Fase 2: analisi e progettazione dell'interfaccia grafica}} \\ \hline 
Analisi dei requisiti dell'interfaccia grafica & 40 \\ \hline
Progettazione dell'interfaccia grafica: visualizzazione dei contenuti & 40 \\ \hline
Progettazione dell'interfaccia grafica: filtraggio dei contenuti & 40 \\ \hline
Progettazione dell'interfaccia grafica: navigazione dei contenuti & 40 \\ \hline
\end{tabular}
\caption{Pianificazione settimanale delle attività}
\label{tab:tesi:stage:pianificazione}
\end{table}

\begin{figure}[ht]
\begin{center}
\includegraphics[width=14.5cm]{gantt.png}
\label{fig:tesi:stage:gantt}
\caption{Diagramma di Gantt}
\end{center}
\end{figure}

\section{Norme di stage}

\subsection{Ambiente di lavoro}
Nel corso dello stage sono stati impiegati diversi strumenti per gestire le attività di progetto e produrre la documentazione prevista.

\begin{table}[ht]
\centering
\begin{tabular}{|l|l|}
\hline
\textsc{Controllo di versione} & \underline{Mercurial} 2.0.2 \\ \hline
\textsc{Editor \LaTeX} & \underline{LaTeXila} 2.4.0 - \underline{gedit} 3.4.0 con \textit{gedit-latex-plugin} \\ \hline
\textsc{Editor UML} & \underline{UMLet} 11.5.1 \\ \hline
\textsc{Foglio elettronico} & \underline{LibreOffice Calc} 3.6 \\ \hline
\textsc{Gestione database} & \underline{MySQL Workbench} 5.2.42 \\ \hline
\textsc{Mockup} & \underline{Pencil} 2.0.2 \\	\hline
\textsc{Pianificazione} & \underline{ProjectLibre} 1.5.1 \\ \hline
\textsc{Repository} & \underline{Bitbucket} \\ \hline
\textsc{Sistema operativo} & \underline{Ubuntu} 12.04 \\ \hline
\end{tabular}
\caption{Configurazione dell'ambiente di lavoro}
\label{tab:tesi:stage:norme:strumenti}
\end{table}

\subsection{Documentazione}
La documentazione è stata redatta in \LaTeX e pubblicata in formato \underline{PDF}.

\paragraph{Struttura}
Ciascun documento presenta una struttura e un formato comuni:
\begin{enumerate}
\item il frontespizio riporta il titolo del documento, l'autore e la data di compilazione;
\item la seconda pagina mostra una sintetica e sommaria presentazione dello scopo del documento;
\item la terza pagina riporta il registro delle modifiche;
\item la quarta pagina mostra l'indice del documento;
\item a seguire è visibile la lista delle figure e delle tabelle presenti nel documento.
\end{enumerate}

\paragraph{Registro delle modifiche}
Il registro delle modifiche tiene traccia della cronologia delle versioni del documento (dalla più recente alla più vecchia), mostrando per ciascuna di esse la data di redazione e una descrizione sintetica delle modifiche apportate.

\paragraph{Versionamento}
A ciascun documento è stato assegnato un numero di versione $x.y$, ove $x$ rappresenta l'ultima \textsc{versione formale}, rivista e approvata dal referente aziendale e disponibile a terze parti interessate (membri del team di progetto, tutor interno), mentre $y$ indica una \textsc{versione preliminare} ad uso interno, eventualmente consultabile dal referente aziendale.

Un incremento del numero di versione secondario $y$ occorre a seguito dell'aggiunta o integrazione dei contenuti o di una revisione informale di tutto o parte del documento.

Un incremento del numero di versione primario $x$ si verifica solamente in seguito alla revisione formale e all'approvazione del documento da parte del referente aziendale.

\paragraph{Nomi dei file}
Il nome assegnato alle versioni preliminari di ciascun documento contiene esclusivamente caratteri alfabetici minuscoli, eventualmente separati mediante il simbolo '-' (trattino). Le versioni formali aggiungono un suffisso, formato dal simbolo '\textunderscore' (trattino basso) e dal numero di versione $x.y$.

\subsection{Modello relazionale}
Il modello relazionale del database è stato realizzato mediante lo strumento adottato dal team di progetto, ossia l'editor \textit{MySQL Workbench}, per facilitare la condivisione e l'integrazione delle informazioni. I file e gli script generati utilizzan la codifica \underline{UTF-8} per garantire la massima compatibilità.

\paragraph{Nomi delle tabelle}
I nomi delle tabelle sono espressi in lingua italiana e contengono solo caratteri alfabetici minuscoli e non accentati, eventualmente separati mediante il simbolo '\textunderscore' (trattino basso).

\paragraph{Nomi degli attributi}
I nomi degli attributi sono espressi in lingua italiana e contengono solo caratteri alfabetici in formato \underline{CamelCase}, ove la lettera iniziale è sempre in minuscolo.

\subsection{Digrammi UML}
Durante l'attività di stage sono stati redatti e inclusi nella documentazione diversi diagrammi dei casi d'uso, dei package e delle classi secondo lo standard \underline{UML} 2.x. Nei paragrafi successivi vengono presentate le linee guida e le convenzioni concernenti la struttura e il formato.

\paragraph{Casi d'uso} La notazione utilizzata per identificare un caso d'uso è così definita:
$$UC.x.y$$
ove:
\begin{itemize}
\item $UC$ è l'abbreviazione di \textit{Use Case} (Caso d'uso);
\item $x \in \left\{1,2,\ldots\right\}$ è il numero identificativo del diagramma cui appartiene il caso d'uso;
\item $y \in \left\{1,2,\ldots\right\}$ è il numero associato al caso d'uso.
\end{itemize}

\paragraph{Package}
I nomi dei package contengono solo caratteri alfabetici minuscoli e non accentati, eventualmente separati dal carattere '-' (trattino).

\paragraph{Classi}
I nomi delle classi sono in formato \textit{CamelCase} e le sottoclassi riportano per esteso o in forma abbreviata l'identificatore della superclasse diretta: nel secondo caso sono presenti - come prefisso - le sole lettere maiuscole, nel medesimo ordine di apparizione.

\subsection{Requisiti funzionali} I requisiti del sistema software sono univocamente identificati mediante la seguente notazione:
$$Rf.x.y$$
ove:
\begin{itemize}
\item $Rf$ è l'abbreviazione di \textit{requisito funzionale};
\item $x \in \left\{ob,de\right\}$ rappresenta il tipo di requisito funzionale (\textit{ob} per obbligatorio, \textit{de} per desiderabili);
\item $y \in \left\{1,2,\ldots\right\}$ è il numero associato ad un requisito.
\end{itemize}

\subsubsection{Tracciamento dei casi d'uso}
Il tracciamento delle dipendenze tra casi d'uso e requisiti software è stato realizzato mediante il foglio elettronico , ove:
\begin{itemize}
\item ciascuna riga rappresenta un requisito del sistema software;
\item ciascuna colonna rappresenta un caso d'uso;
\item ciascuna cella contiene il carattere 'X' se esiste una relazione di dipendenza tra il caso d'uso e il requisito, altrimenti è vuota.
\end{itemize}

Per ciascuna riga e colonna viene impiegata una semplice formula per asserire la completezza e la necessità della matrice dei requisiti:  
\begin{center}
\texttt{CONTA.SE(A:Z;``X'')}   
\end{center}
ove:
\begin{itemize}
\item \texttt{A:Z} corrisponde all'intervallo di celle di una singola riga o colonna;
\item \texttt{"X"} rappresenta il pattern da cercare (nello specifico, una stringa di lunghezza unitaria);
\item \texttt{CONTA.SE} è una funzione a due argomenti (intervallo di celle, pattern) che restituisce il numero di celle nell'intervallo indicato contenenti una o più occorrenze del pattern specificato.
\end{itemize}

\paragraph{Completezza} Per ogni colonna, se la formula restituisce un valore pari a 0 (zero) sta ad indicare che il requisito utente non è soddisfatto da alcun requisito software.

\paragraph{Necessità} Per ogni riga, se la formula restituisce un valore pari a 0 (zero) sta ad indicare che il requisito software corrispondente è superfluo.

%--------
% SECTION
%--------
\section{Fase 1: criterio di classificazione}
\label{sec:tesi:stage:fase-1}
Il patrimonio di conoscenza della piattaforma è garantito essenzialmente dai contenuti pubblicati dagli utenti ed arricchito dal loro valore informativo: ciascuno di essi, a prescindere dalla forma (testo, immagini, audio, video, \ldots) o dalla classe (domanda, discorso, evento, recensione, \ldots), condivide delle informazioni inerenti uno o più elementi del dominio tematico della piattaforma.

\begin{figure}[ht]
\begin{center}
 \includegraphics{placeholder.png}
 \label{fig:tesi:stage:classificazione:serbatoio-contenuti}
 \caption{Contenuti informativi e conoscenza}
\end{center}
\end{figure}

\paragraph{Contenuti informativi}
Allo state attuale, la piattaforma si limita ad essere un serbatoio di \textsc{contenuti informativi} disaggregati, privo degli strumenti per classificare e catalogare il sapere in essa custodito conferendovi una struttura ordinata, una sorta di indice enciclopedico in grado di facilitarne la ricerca, il reperimento e la consultazione.

Ciascun contenuto rappresenta - dal punto di vista conoscitivo - una collezione di frammenti di informazioni, ciascuno dei quali contribuisce ad arricchire la conocenza relativa a qualche \textsc{entità}, oggetto di discussione all'interno del dominio della piattaforma.

\begin{figure}[ht]
\begin{center}
 \includegraphics{placeholder.png}
 \label{fig:tesi:stage:fase-uno:contenuti-informativi}
 \caption{Valore informativo di un contenuto}
\end{center}
\end{figure}

\paragraph{Dominio conoscitivo}
L'insieme di entità definite - in un certo istante - all'interno della piattaforma ne costituisce il \textsc{dominio della conoscenza} (di seguito per brevità \textsc{dominio}), in maniera analoga a quanto accade con i lemmi di un'enciclopedia. Ciascun frammento di informazione presente in un contenuto è concettualmente associabile e riferibile ad un'entità di tale dominio.

L'obiettivo primario del nuovo criterio di classificazione consiste dunque nel modellare tale dominio e le relazioni esistenti tra le relative entità ed i contenuti informativi.

\begin{figure}[ht]
\begin{center}
 \includegraphics{placeholder.png}
 \label{fig:tesi:stage:fase-uno:dominio-conoscenza}
 \caption{Dominio di conoscenza della piattaforma}
\end{center}
\end{figure}

\subsection{Entità}  
Le \textsc{entità} $d_i \in D$ del dominio rappresentano elementi concreti (luoghi, persone, eventi, \ldots) o astratti (concetti, \ldots) a cui afferiscono i contenuti. Ciascuna di esse rappresenta un lemma dell'enciclopedia del sapere disponibile presso la piattaforma e - pur avendo un preciso valore semantico - dev'essere identificata sul piano sintattico mediante un'\textsc{etichetta} $e_j \in E$.

\begin{equation}
e_j = f(d_i)
\end{equation}

\paragraph{Ambiguità sintattica}
Gli utenti possono riferirsi ad un'entità $d_i$ con termini o espressioni differenti ($e_{i,j} \in E_i$): tale ambiguità non può essere ignorata o trascurata e impone di considerare la relazione tra l'entità e le etichette con cui può essere riferita di tipo uno-a-molti. Il criterio di classificazione dev'essere quindi in grado di esprimere il fatto che tali etichette rappresentino sinomini di una stessa entità.

\begin{figure}[ht]
\begin{center}
\includegraphics{placeholder.png}
\label{fig:tesi:stage:fase-uno:ambiguita-sintattica-entita}
\caption{Ambiguità sintattica di un'entità}
\end{center}
\end{figure}

\paragraph{Etichette duplicate}
Allo stesso tempo occorre impedire la proliferazione di etichette duplicate, ossia equivalenti sul piano semantico ma sintatticamente differenti. Un fenomeno simile avrebbe inevitabili ripercussioni sull'efficacia del criterio di classificazione e sull'efficienza della ricerca: reperire tutte e sole le informazioni inerenti una certa entità richiederebbe infatti di individuare tutte le etichette con cui possa essere riferita e cercare riscontri per ciascuna di esse nei contenuti pubblicati.

Per incrementare l'efficienza di catalogazione e ricerca dei contenuti risulta conveniente che l'entità sia identificata univocamente nei contenuti informativi, a prescindere dalla specifica etichetta utilizzata.

\paragraph{Sintassi e semantica}
Il fattore essenziale consiste nel mantenere separata la componente semantica (il dominio delle entità) da quella sintattica (il dizionario delle etichette): solo così è possibile stabilire un associazione naturale e diretta tra i contenuti e le entità, a prescindere dalle etichette utilizzabili per riferirla.

\begin{figure}[ht]
\begin{center}
\includegraphics{placeholder.png}
\label{fig:tesi:stage:fase-uno:entita-sintassi-semantica}
\caption{Sintassi e semantica di un'entità}
\end{center}
\end{figure}

Ciò non toglie la necessità di individuare un'etichetta, che permetta di esprimere e comunicare l'entità cui si fa riferimento: per soddisfare tale condizione si individua - per ciascuna entità - un'\textsc{etichetta primaria} $e_{i,0}$, che la identifica univocamente nell'ambito della piattaforma, mentre le restanti (\textsc{etichette secondarie}) ne vengono considerate sinonimi.

\begin{equation}
e_{i,0} = f'(d_i)
\end{equation}

\paragraph{Relazioni}
Ciascun lemma enciclopedico contiene spesso riferimenti ad altre voci, che trattano temi specifici, particolari o ad esso affini: ci si aspetta che tali relazioni si possano replicare anche nel dominio della piattaforma, evidenziando possibili legami molti-a-molti tra le entità.

\begin{figure}[ht]
\begin{center}
\includegraphics{placeholder.png}
\label{fig:tesi:stage:fase-uno:entita-relazioni}
\caption{Relazioni tra entità}
\end{center}
\end{figure}

Il dominio assume la struttura e le caratteristiche di un grafo orientato ove:
\begin{itemize}
\item ciascun nodo rappresenta un entità, identificata dalla relativa etichetta primaria;
\item ciascun arco uscente identifica un'entità riferita;
\item ciascun arco uscente identifica un'entità da cui l'attuale è riferita.
\end{itemize}

\subsection{Etichette}
Un'\textsc{etichetta} $e_j$ rappresenta la forma sintattica - una stringa di lunghezza variabile - mediante la quale gli utenti ed il sistema identificano un'entità del dominio. L'insieme di etichette definite in un certo istante rappresenta il \textsc{dizionario} $E$ della piattaforma.

\paragraph{Accezioni}
Analogamente ad un lemma enciclopedico, un'etichetta $e_j$ può risultare semanticamente ambigua, essendo caratterizzata da svariate \textsc{accezioni} $a_{j,k}$, ciascuna delle quali assume un significato e si riferisce ad un'entità differenti. Sotto questo aspetto, la natura tematica della piattaforma dovrebbe contribuire a limitare il numero medio di accezioni per ciascuna etichetta.

\begin{figure}[ht]
\begin{center}
\includegraphics{placeholder.png}
\label{fig:tesi:stage:fase-uno:etichette-accezioni}
\caption{Accezioni di un'etichetta}
\end{center}
\end{figure}

L'etichetta $e_j$ identifica dunque entità del dominio distinte, a seconda dell'accezione $a_{j,k}$ considerata: la distinzione tra etichetta primaria e secondaria si trasferisce a livello delle accezioni, ciascuna delle quali puà essere \textsc{chiave} o \textsc{sinonimica}.

\paragraph{Accezione chiave}
L'accezione chiave ($a_{j,0}$) indica che la relativa etichetta identifica univocamente l'entità corrispondente: per tale ragione - in ogni istante - a ciascuna entità dev'essere associata una e una sola etichetta primaria. Qualora si rimuova un'accezione chiave di un'etichetta, occorre individuare una nuova etichetta primaria per l'entità riferita.

\paragraph{Accezione sinonimica}
L'accezione sinonimica ($a_{j,1},\ldots,a_{j,\left|A_j\right|}$) indica che l'etichetta associata viene utilizzata come sinonimo dell'entità corrispondente. La relazione uno-a-molti tra entità ed etichette si traduce dunque in:
\begin{itemize}
\item una relazione uno-a-molti tra le etichette e le accezioni:
\begin{equation}
e_j = g(a_{j,k})
\end{equation}
\item una relazione uno-a-molti tra le entità e le accezioni:
\begin{equation}
d_i = h(a_{j,k})
\end{equation}
\end{itemize}

\paragraph{Sinonimi}
Sebbene le entità siano identificate univocamente dalle corrispondenti etichette primarie, è assai utile includere e conservare nel dizionario anche i sinonimi (etichette secondarie) note o utilizzate dagli utenti per garantire maggiore copertura sintattica, aumentando la probabilità che i termini o le espressioni utilizzati in futuro per riferire una certa entità siano riconosciuti in quanto gia presenti nel dizionario.

\paragraph{Ricerca}
L'utente alla ricerca di informazioni su un particolare tema si trova a cercare delle etichette che siano attinenti al tema di interesse. Per evitare la proliferazione di etichette semanticamente identiche e sintatticamente simili, si fissa uno standard per il formato, che normi la capitalizzazione delle lettere, la gestione degli spazi, \ldots al fine di uniformare la struttura sintattica delle etichette.

L’osservanza e l’adesione a tali regole da parte delle etichette inserite può essere accertata automaticamente, provvedendo - ove necessario - ad apportare le opportune correzioni per adeguarle allo standard, e rappresenta un requisito essenziale per garantire la consistenza del dizionario della piattaforma.

\subsection{Contenuti}  
L'obiettivo primario del criterio di classificazione consiste nel tenere traccia delle entità riferite all'interno di ciascun contenuto per consentire agli utenti di reperire facilmente informazioni ad esse relative.

\paragraph{Valore informativo}
Ciascun contenuto in generale reca con sé informazioni riguardanti diverse entità: ne consegue che la relazione tra entità e contenuti sia di tipo molti-a-molti (\texttt{n:m}). Aver preservato la distinzione tra entità (semantica) ed etichette (sintassi) consente di identificare in maniera chiara e univoca nell'intera piattaforma ciò a cui si fa riferimento nei vari contenuti, a prescindere dai termini o dalle espressioni linguistiche impiegati per indicarli.

\paragraph{Catalogazione}
L'assegnazione di un'etichetta ad un contenuto è un processo che traduce un ingresso sintattico dell'utente (l'etichetta) in un'uscita semantica (l'entità), che viene effettivamente e concretamente assegnata al contenuto.

Una volta individuata l'etichetta, qualora presenti accezioni multiple spetta all'utente indicare quella corrispondente all'entità intesa. Se l'accezione scelta risultasse sinonimica, per identificare l'entità da assegnare al contenuto essa verrebbe automaticamente sostituita dalla corrispondente primaria.

Sebbene si sia scelto di riferire - nei contenuti pubblicati - ciascuna entità sempre e solo con l'etichetta primaria, per ragioni di semplicità ed efficienza (l'etichetta primaria è un'informazione strettamente correlata all'entità), nulla impedirebbe di assegnare ai contenuti le etichette scelte dagli utenti, a prescindere dal fatto che siano primarie o secondarie. Ciò comporterebbe tuttavia la necessità di associare al contenuto non solo l'entità, ma anche - tra le possibili - l'etichetta scelta per indicarla.

\subsection{Utente}
Le scelte progettuali illustrate nelle sezioni precedenti sono state valutate secondo i principali approcci e obiettivi con cui l'utente esplora i contenuti pubblicati nella piattaforma, arrivando ad individuare diversi casi d'uso fondamentali. Si rivela particolarmente utile in tale frangente la metafora enciclopedica, che accosta il nuovo criterio di classificazione ad un indice enciclopedico, giacché le principali differenze riscontrabili tra i casi d'uso sono assimilabili agli scopi e alle modalità con cui si consulta un'enciclopedia del sapere. 

\paragraph{Ricerca per temi}
La ricerca per temi consiste nell'esplorare i contenuti informativi della piattaforma a partire dalle entità, che identificano gli argomenti di discussione, in maniera analoga alla consultazione dei lemmi di un'enciclopedia: si procede da un lemma all'altro sino a trovare quella di interesse facendosi guidare dai riferimenti tra le voci. Nella piattaforma ciò si traduce in un'esplorazione libera del grafo orientato delle entità, ciascuna identificata dalla relativa etichetta primaria. 

\paragraph{Ricerca per etichette}
La ricerca per etichette si basa sull'inserimento di alcune parole chiave, che agli occhi dell'utente identificano l'informazione cui è interessato e hanno maggiore probabilità di essere associate ai contenuti: maggiore è il numero o la specificità dei termini immessi, più la ricerca risulta circostanziata.

Nella piattaforma, una volta riconosciuta un'etichetta inserita occorre identificare il valore semantico (l'entità) attribuitole dall'utente, che è associato ad un'accezione dell'etichetta stessa, per giungere ad identificare l'entità cercata.

Ove la ricerca si limitasse a considerare l’etichetta inserita, i risultati potrebbero risultare parziali, contemplando i soli contenuti in cui l’entità sia riferita dalla specifica etichetta, o addirittura non pertitenti, includendo contenuti ove l'etichetta è presente in un'accezione differente. Un ulteriore vantaggio da considerare riguarda la dimensione del dizionario e del dominio: sapendo che ciascuna accezione rappresenta un'etichetta distinta per una certa entità possiamo esprimere il numero medio di etichette associate a ciascuna entità come
\begin{equation}
\alpha = \frac{\sum{\left|A_j\right|}}{\left|D\right|}
\end{equation}
Assegnando ai contenuti un'etichetta qualsiasi si aumenterebbe mediamente di una costante moltiplicativa $\alpha$ la complessità della ricerca, dovendola ripetere per ciascuna delle $\alpha$ etichette invece che per la sola entità corrispondente.\footnote{Si assume per ipotesi che la ricerca verifichi - per ciascun contenuto - quali termini cercati (entità o etichette) siano presenti, uno per volta, e che la complessità computazionale sia equivalente in entrambi i casi.}

La rilevanza di un contenuto informativo rispetto ai criteri di ricerca è quindi determinata dalla percentuale di corrispondenza delle entità cercate: maggiore è il numero di entità riscontrate nel contenuto, maggiore sarà approssimativamente la rilevanza attribuita nel contesto della ricerca.

Se $E_s$ è l'insieme delle etichette cercate e $E_c$ l'insieme delle etichette assegnate a ciascun contenuto si possono distinguere tre casi principali:
\begin{description}
\item[Corrispondenza completa:] $E_s \subseteq E_c$ \hfill \\
Al contenuto risultano assegnate tutte le etichette richieste dall'utente (massima attinenza).
\item[Corrispondenza parziale:] $E_s \cap E_c \neq \emptyset$ \hfill \\
Al contenuto risulta assegnata parte delle etichette richieste dall'utente (media attinenza).
\item[Nessuna corrispondenza:] $E_s \cap E_c = \emptyset$\hfill \\
Al contenuto non risulta assegnata alcuna etichetta richiesta dall'utente (attinenza nulla).
\end{description}

\paragraph{Ricerca per affinità}
La ricerca di contenuti affini consiste nell'identificare, a partire da un contenuto dato, quelli aventi il maggior numero di entità in comune: in questo scenario valgono le medesime considerazioni fatte nella sezione precedente, previa sostituzione di $E_s$ con l'insieme delle etichette assegnate al contenuto corrente.

\subsection{Modello relazionale}

\begin{figure}[ht]
\begin{center}
\includegraphics[width=14.7cm]{modello-er.png}
\label{fig:tesi:stage:er:modello}
\caption{Modello relazione del criterio di classificazione}
\end{center}
\end{figure}

La tabella \textsf{contenuti} appartiene al modello relazionale della piattaforma ed è stato integrato per evidenziare alcune relazioni fondamentali, illustrate di seguito. Le restanti tabelle sono concettualmente organizzate in tre \textit{package} (\textsf{contenuti}, \textsf{semantica} e \textsf{sintassi}) per chiarirne il ruolo all'interno del sistema di classificazione.

\paragraph{Entità}
Ciascuna entità viene rappresentata mediante la tabella (\textsf{entita}) e si caratterizza essenzialmente per:
\begin{description}
\item[tipo] \hfill \\
Chiave esterna per la tabella \textsf{tipi\textunderscore entita}, indica se si tratti di un luogo, un evento, una persona, un concetto astratto, \ldots. 
\item[etichetta] \hfill \\
Chiave esterna per la tabella \textsf{etichette}, rappresenta l'etichetta primaria associata all'entità.
\end{description}

La relazione di tipo molti-a-molti tra le entità è modellata mediante la tabella \textsf{gerarchia}, che contiene due chiavi esterne verso la medesima tabella \textsf{entita}. Nella derivante struttura a grafo ciascuna istanza della classe \textsf{gerarchia} rappresenta un arco, di cui \textsf{padre} è il nodo uscente e \textsf{figlio} quello entrante.

\paragraph{Etichette}
Le etichette sono modellate dalla tabella \textsf{etichette} e hanno come chiave primaria il campo \textsf{valore}, che rappresenta la stringa corrispondente.

La scelta di utilizzare la stringa come chiave primaria risponde alla naturale identificazione dell'etichetta nella sequenza di caratteri corrispondente, costituisce una garanzia contro la presenza di duplicati e consente di recuperare il valore dell'etichetta primaria di un'entità senza dover effettuare un'operazione di \textit{join} tra le tabelle \textsf{entita} ed \textsf{etichette}.

\paragraph{Accezioni}
Un'accezione rappresenta un legame univoco tra un'etichetta e un'entità e può essere \textsc{chiave} o \textsc{sinonimica}. Essa non presenta attributi propri significativi, ma prevede tre vincoli referenziali:
\begin{enumerate}
\item a ciascuna entità è associata una e una sola \textsc{accezione chiave} (relazione uno-a-uno), che identifica l'etichetta primaria;
\item a ciascuna entità sono associate $0\ldots n$ \textsc{accezioni sinonimiche} (relazione uno-a-molti), che rappresentano i sinonimi dell'etichetta primaria;
\item ciascuna etichetta possiede $0\ldots n$ \textsc{accezioni} (relazione uno-a-molti).
\end{enumerate}

\begin{figure}[ht]
\begin{center}
\includegraphics{placeholder.png}
\label{fig:tesi:stage:er:accezioni}
\caption{Modello ad oggetti delle accezioni}
\end{center}
\end{figure}

Ne consegue che sia la superclasse (\textsf{accezioni}) sia le sottoclassi (\textsf{accezioni\textunderscore chiave} e \textsf{accezioni\textunderscore sinonimiche}) presentano dei vincoli referenziali; in particolare, i primi due scostringono a distinguere - dal punto di vista dell'entità - tra accezioni chiave e sinonimiche.

Per modellare tale scenario sono stati presi in considerazione tre possibili approcci risolutivi:
\begin{description}
\item[Tabella unica] \hfill \\
La tabella unica ben si adatta a gestire l'assenza di attributi propri per le sottoclassi e ad esprimere i vincoli referenziali che coinvolgono la superclasse, ma non è in grado di esprimere e adeguatamente rappresentare quelli coinvolgenti le sottoclassi.
\item[Partizionamento orizzontale] \hfill \\
Il partizionamento orizzontale riesce a modellare i vincoli referenziali delle sottoclassi, ma non quello della superclasse, e genera due classi aventi i medesimi attributi.
\item[Partizionamento verticale] \hfill \\
Il partizionamento verticale consente di modellare correttamente tutti e tre i vincoli referenziali, relativi sia alla superclasse sia alle sottoclassi. Tuttavia si rende più complesso modificare il tipo di un'accezione e si introduce l'esigenza di un'operazione \textit{join} per recuperare la lista completa delle accezioni, pur non possedendo le sottoclassi attributi propri.
\end{description}

La decisione finale di adottare la soluzione della tabella si accompagna ad alcune osservazioni:
\begin{itemize}
\item la distinzione tra etichette chiave e sinomimiche ha rilevanza essenzialmente dal punto di vista delle entità;
\item il vincolo referenziale tra \textsf{etichette} ed \textsf{accezioni} suggerisce che la distinzione menzionata al punto precedente sia irrilevante dal punto di vista delle etichette, ragion per cui risulta utile mantenere tutte le accezioni nella medesima tabella.\footnote{Si consideri ad esempio il caso d'uso della ricerca di un'entità mediante un'etichetta, in cui occorre recuperare la lista completa delle accezioni relative all'etichetta stessa.}
\item le sottoclassi non hanno attributi propri, per cui il partizionamento verticale e orizzontale rappresentano soluzioni inadeguate o carenti;
\end{itemize}

Il compromesso viene raggiunto eliminando qualsiasi riferimento al tipo dell'accezione nella classe \textsf{accezioni} e modellando la relazione uno-a-uno tra le entita e le relative etichette primarie mediante una chiave esterna nella classe \textsf{entita}, ossia \textsf{entita.etichetta}, che identifica la corrispondente etichetta primaria nella tabella \textsf{etichette}. Così facendo è possibile esprimete tutti i vincoli referenziali sopra esposti senza definire le sottoclassi.

\paragraph{Contenuti}
Il vincolo referenziale tra i contenuti e le entità rappresenta l'essenza del criterio di classificazione, consentendo di classificare e catalogare le informazioni disponibili nella piattaforma. La relazione molti-a-molti tra \textsf{contenuti} ed \textsf{entita} viene modellata aggiungendo la nuova classe \textsf{classificazione}, che tiene traccia delle entità associate a ciascun contenuto.

%--------
% SECTION
%--------
\section{Fase 2: interfaccia grafica}
\label{sec:tesi:stage:fase-2}

\subsection{Analisi dei requisiti}

\subsection{Progettazione}
