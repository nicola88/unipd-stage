\chapter{Valutazioni finali}
\label{ch:tesi:valutazioni}

\section{Obiettivi}
Nel corso dello stage sono state svolte due attività correlate, ma radicalmente diverse sul fronte del tipo di impegno e di competenze richieste:
\begin{enumerate}
  \item lo sviluppo di un criterio di classificazione aggiuntivo per catalogare il sapere custodito nei contenuti pubblicati dagli utenti nella piattaforma;
  \item l'analisi e la progettazione di un'interfaccia grafica per la consultazione dei risultati di una ricerca sui contenuti informativi.
\end{enumerate}

L'esito delle attività è stato nel complesso soddisfacente: gli obiettivi fissati sono stati raggiunti e tutti i vincoli o requisiti voluti dal committente sono stati esauditi.

\subsection{Criterio di classificazione}

\subsection{Interfaccia grafica}

\section{Competenze professionali}
Le attività di stage si inseriscono nell'ambito di un percorso di collaborazione con l'azienda \textit{Sintesi Sas}, iniziato in occasione dell'iniziativa \textit{Mimprendo} promossa da \textit{Confindustria Padova}, che mi ha permesso di seguire il percorso evolutivo del progetto S(L)S sin dai primi albori ed ha rappresentanto un'opportunità di crescita professionale molto rilevante.

Sin dalle attività precedenti e in particolare nel corso dello stage ho potuto apprezzare gli enormi benefici - in termini di competenze professionali acquisite - derivanti dall'operare all'interno di un progetto caratterizzato da un forte carattere multidisciplinare: il team di progetto include infatti figure provenienti da svariati settori professionali (psicologia, sociologia, marketing, economia, informatica, ingegneria, \ldots), che collaborano attivamente e a stretto contatto integrando le reciproche conoscenze e competenze.

Nel corso dell'attività di stage ho avuto occasione di confrontarmi con queste figure, da cui ho appreso informazioni che mi hanno consentito di svolgere le attività previste nel modo migliore e che mi hanno fatto comprendere come il successo di un progetto non possa in alcun modo prescindere dal contributo di competenze professionali diverse.

L'analisi, la progettazione e lo sviluppo della piattaforma S(L)S non si limitano a considerare i soli aspetti tecnici o tecnologici nella concezione e nella valutazione delle soluzioni da adottare nell'intero ciclo di vita del software, ma cercano di comprendere ed intepretare le esigenze degli utenti e le dinamiche sociali. In questa piattaforma di socializzazione e di condivisione della conoscenza la tecnologia è diventata non il fine, bensì il mezzo attraverso il quale si cerca di concretizzare un certo tipo di esperienza virtuale (e non).

\subsection{Criterio di classificazione}
La progettazione del criterio di classificazione ha richiesto nella fase iniziale dello stage un'attività di ricerca riguardante lo stato dell'arte dei sistemi di classificazione attualmente adottati nelle principali piattaforme di condivisione di contenuti:
\begin{itemize}
  \item blog (\textit{Drupal}, \textit{Wordpress});
  \item forum;
  \item social network (\textit{Facebook}, \textit{Twitter}).
\end{itemize}

I risultati di questa analisi hanno consentito di conoscere i principali meccanismi di catalogazione delle informazioni, di evidenziarne le principali analogie e differenze e di coglierne i punti di forza e di debolezza: a partire da queste informazioni ho inteso affrontare lo sviluppo di un criterio di classificazione, che riuscisse a coniugare - nella soluzione più semplice e adatta alla piattaforma \textit{Social (Life) Shuttle} - gli aspetti essenziali riuscendo al contempo a superarne i limiti evidenziati.

Durante la fase di integrazione del criterio di classificazione nel modello relazionale della piattaforma le competenze acquisite in ambito universitario mi hanno permesso di individuare soluzioni semplici ma efficaci, che sono state accolte e giudicate positivamente dal team di progetto. Le principali difficoltà incontrate riguardano l'integrazione della propria soluzione nel modello relazionale esistente in maniera coerente e collaborativa, essendo diverse persone impegnate, seppur su fronti diversi, alla modifica e all'aggiornamento dello stesso.

\subsection{Interfaccia grafica}
Le attività previste per la seconda fase dello stage sono state caratterizzate solo in minima parte da attività di ricerca, integrata in molti aspetti da indagini di carattere sociologico. Queste, seppur preliminari, hanno rappresentato un contributo cruciale nel processo di identificazione e comprensione delle variegate esigenze di utenti con profili esperienziali diversificati e di valutazione delle scelte progettuali effettuate.

La maggior parte del tempo a disposizione è stato speso per effettuare l'analisi dei requisiti, che ha permesso di delineare i casi d'uso degli utenti e individuare alcuni requisiti fondamentali: in questa fase si è rivelata decisiva per riuscire a svolgere le attività previste in modo da rispettare il piano di lavoro un'organizzazione del lavoro e una metodologia acquisiti e maturati durante la carriera universitaria.

La documentazione prodotta, in particolare, è stata giudicata assai favorevolmente per la completezza e l'organicità con cui sono stati presentati e illustrati in dettaglio i risultati del lavoro svolto nell'arco delle varie settimane.

\section{Stage e università}
%- team ampio e multidisciplinare
Sebbene la formazione universitaria mi abbia consentito di superare abbastanza agevolmente le sfide e le difficoltà incontrate lungo il percorso di stage, grazie alle conoscenze e alle metodologie di lavoro acquisite, il principale fattore di novità e di straniamento riscontrato nell'ambito del progetto S(L)S è rappresentato dal taglio fortemente multidisciplinare, a cui tutti sono chiamati a concorrere ed attingere e che costringe a rivedere un modello di pensiero ed una \textit{forma mentis} principalmente preparate e abituate a focalizzarsi sull'aspetto principalmente tecnico e tecnologico, avente maggior pertinenza ed interesse rispetto alle figure professionali degli informatici.

%- apertura alla caratterizzazione multidisciplinare della figura professionale dell'informatica
Sin dalle prime comunicazioni con il futuro referente aziendale, mi è apparso chiaro come l'attitudine a pensare in termini multidisciplinari fosse necessaria per comprendere lo spirito e gli obiettivi del progetto: maturata nel corso dei mesi, grazie al dialogo e alla collaborazione con gli altri membri del team di progetto, l'approccio multidisciplinare rappresenta un elemento cruciale per integrarsi con successo in un progetto simile e comprenderne a fondo le reali esigenze ed il genuino spirito.

- il percorso continua a livello professionale