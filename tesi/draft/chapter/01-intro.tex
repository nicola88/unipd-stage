\chapter{Introduzione}
\label{ch:tesi:intro}

\section{Contenuti}
Il presente documento costituisce una relazione dettagliata in merito all'attività di stage svolta dallo studente Nicola Moretto presso l'azienda \textit{Sintesi Srl}. I contenuti sono organizzati nei seguenti capitoli:
\begin{description}
  \item[\nameref{ch:tesi:intro}] \hfill \\
  Il primo capitolo illustra brevemente la struttura del documento e le convenzioni tipografiche utilizzate.
  \item[\nameref{ch:tesi:azienda}] \hfill \\
  \ldots
  \item[\nameref{ch:tesi:progetto}] \hfill \\
  \ldots
  \item[\nameref{ch:tesi:stage}] \hfill \\
  \ldots
  \item[\nameref{ch:tesi:valutazioni}] \hfill \\
  \ldots
  \item[\nameref{ch:appendice:glossario}] \hfill \\
  \ldots
\end{description}

\section{Convenzioni tipografiche}
Al fine di agevolare la consultazione del documento, sono state adottate alcune convenzioni tipografiche illustrate di seguito.

\paragraph{Glossario} Gli acronimi, le abbreviazioni, i nomi propri e i termini specialistici contenuti nel presente documento sono illustrati nel \textit{\nameref{ch:appendice:glossario}}, consultabile in appendice, al fine di agevolare la lettura e la comprensione degli argomenti trattati.	La prima occorrenza di ciascun termine o espressione presente nel glossario è riconoscibile per la \underline{sottolineatura}.

\paragraph{Terminologia} I termini propri o di provenienza straniera divenuti di uso corrente nella lingua italiana sono evidenziati in \textit{corsivo}, mentre le parole o espressioni di particolare rilevanza o significato nel presente contesto sono evidenziate in \textsc{maiuscoletto}.

\paragraph{Codice e formule} I nomi di tabelle, classi, package, \ldots sono riportati con un carattere di tipo \textsf{sans serif}, mentre i frammenti di codice o formule sono riconoscibili per l'impiego di un carattere a \texttt{spaziatura fissa}.
