\chapter{Progetto}
\label{ch:tesi:progetto}

\section{Genesi}
\label{sec:progetto:genesi}
L'idea della piattaforma \textit{Social (Life) Shuttle} nasce nel 2010 da un progetto concepito per creare una comunità virtuale destinata agli artisti sconosciuti accessibile in mobilità mediante un'applicazione dedicata, \textit{ArtYR}.

Nello stesso periodo una consulenza nell'ambito dei sistemi informativi terrioriali ad un'azienda di Bolzano conduce allo sviluppo di un'innovativa piattaforma software: si tratta di un sistema informativo territoriale in cui l'erogazione di informazioni turistiche è integrata con la vendita di servizi turistici.

Il progetto evolve - grazie alla partecipazione di Comuni, Province e Regioni - in una rete tematica di agenzie di viaggio con un'identità comune e finalizzata alla fusione dei sistemi informativi distrettuali e di vendita.

L'architettura di \textit{Social (Life) Shuttle} ne trae ispirazione, integrando tre componenti differenti:
\begin{enumerate}
	\item \textit{business}: vendita diretta di prodotti alla clientela;
	\item \textit{social}: creazione e sviluppo delle relazioni sociali attraverso la condivisione di informazioni e conoscenza;
	\item sistema di erogazione di informazioni turistiche e territoriali.
\end{enumerate}

\section{Reti sociali}
\label{sec:tesi:progetto:reti-sociali}
Il modello sociologico di \underline{rete sociale} non ha attualmente riscontro presso le piattaforme web di condivisione dei contenuti (blog, forum, \ldots) o i \textit{social network} (Facebook, Twitter, \ldots), che si limitano a cconsiderarne o concretizzarne singoli aspetti.

Nelle moderne reti sociali è infatti assente l'incentivo alla condivisione e distribuzione della conoscenza, fattore cruciale per l'aggregazione fisica dei membri delle comunità, intese come aggregazioni che si formano intorno ed attraverso la manifestazione di interesse nei confronti di uno specifico tema di dialogo o discussione, che attraversa la sfera individuale, intima e personale dei suoi membri.

Il progetto \textit{Social (Life) Shuttle} rappresenta una nuova generazione delle piattaforme di socializzazione, in cui il web diventa solamente un canale di condivisione e un serbatoio della conoscenza generata dalla dialettica tra persone e vengono ad essere integrati i canoni classici di \textit{blog}, \textit{forum}, \textit{social network} e \textit{media} .

Una relazione sociale nata e costruita su un interesse comune stravolge l'attuale paradigma delle reti sociali virtuali, in cui il legame nasce a prescindere dalla presenza di interessi comuni o informazioni da condividere, e favorisce l'incontro tra persone aventi esperienze simili frutto di tali interessi condivisi. Ove l'esperienza riguardi anche beni o prodotti, la componente business inende offrire ai membri la possibilità di interagire con i produttori, anch'essi attori della comunità.

L'architettura di \textit{Social (Life) Shuttle} consente di declinare la piattaforma in innumerevoli varianti, applicabili ai temi più svariati: al momento sono in fase di sperimentazione per il mondo del vino, il cibo biologico, l'arte commercializzabile e l'attività di ricerca e progettazione collaborativa. 

\section{Architettura}
\label{sec:tesi:progetto:architettura}
Tale piattaforma presenta numerose aspetti che la differenziano dalla concorrenza attuale:
\begin{itemize}
	\item profonda integrazione degli aspetti \textit{social} e \textit{business};
	\item nessuna distinzione tra creatori e fruitori dei contenuti (ciascun membro può condividere le proprie esperienze, segnalare eventi, pubblicare articoli critici, \ldots);
	\item l'autorevolezza di ciascun membro della comunità si rafforza o si indebolisce a seconda della qualità dei contenuti pubblicati, dei giudizi degli altri membri o di altri criteri di valutazione;
	\item lo sfruttamento delle tecnologie e dei dispositivi mobili per favorire l'instaurazione o la coltivazione di relazioni al di fuori dell'ambito virtuale della piattaforma (partecipazione ad eventi, raccolta e condivisione di informazioni geolocalizzate, \ldots).
\end{itemize}  

% definizione di rete sociale: Granovetter.
Per quanto concerne le attività di stage, due aspetti della piattaforma assumono particolare rilevanza: i contenuti informativi e i relativi criteri di classificazione.

\subsection{Contenuti informativi}
\label{sec:progetto:contenuti}
I contenuti informativi rappresentano lo strumento essenziale per la condivisione delle esperienze e della conoscenza intorno al tema specifico della piattaforma.

L'individuazione di un insieme di classi di contenuti adatti a esprimere in una forma strutturata le informazioni è frutto di un lungo processo, che ha  tiene conto essenzialmente delle forme di espressione e di comunicazione tipiche nella vita quotidiana, che a loro volta rispecchiano l'intenzione comunicativa dell'azione compiuta e delle parole espresse da un singolo individuo.

- distinzione tra natura e formato del contenuto
	- natura: forme espressive nello scambio dialettico quotidiano per favorire identificazione immediata del formato appropriato
	- formato: supporti del web (audio, video, testo, chat, ...) [analisi comparativa] per stabilire i modi classici e preferiti dagli utenti per condividere informazioni
- non imporre vincoli alla liberà creativa degli utenti
- facilitare la catalogazione

%\paragraph{La metafora dei Lego}
%$$ contenuto informativo = tipo di contenuto + elementi di un contenuto $$

\subsection{Criteri di classificazione}
\label{sec:progetto:classificazione}
