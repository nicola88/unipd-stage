\documentclass[11pt,a4paper,headsepline,hidelinks]{scrreprt} % KOMA-Script
\usepackage[italian]{babel}
\usepackage[utf8]{inputenc}
\usepackage[T1]{fontenc}
\usepackage{graphicx}
\usepackage{hyperref}
\usepackage{amsfonts}
\pagestyle{headings}

% MACRO
\def \azienda{\textit{Sintesi Srl}}
\def \sls{\textit{Social (Life) Shuttle}}

\begin{document}
    \title{Relazione di stage - Bozza}
%	\subject{Analisi e progettazione di un'interfaccia grafica per la consultazione dei contenuti informativi in una piattaforma web tematica}
    \author{Nicola Moretto (matr. 578258)}
    \date{\today}

    \maketitle

	\tableofcontents

	\listoffigures
	\begingroup
	\let\clearpage\relax
	\listoftables
	\endgroup

%	\part{Contesto}

	% CHAPTER
	\chapter{L'azienda}
	L'attività di stage si è svolta presso l'azienda \azienda{}, operante nel settore IT (\textit{Information Technology}) grazie alla vendita e assistenza di software gestionali proprietari rivolti ad aziende attive nel settore alberghiero. Si tratta di una piccola realtà imprenditoriale a clientela nazionale con sede unica a Mestre (VE), la cui gestione e amministrazione è affidata al solo fondatore, che ha assunto il ruolo di tutor esterno e referente aziendale per l'intero arco di svolgimento dello stage.

	Le decisioni assunte e le informazioni prodotte nel corso dello stage sono state condivise, discusse e approvate dal suddetto referente, al quale si è fatto riferimento per domande, dubbi o questioni emersi nel corso delle attività.

	% CHAPTER
	\chapter{Il progetto}
	L'attività di stage svolta presso l'azienda \azienda{} si inserisce nel quadro di un progetto complesso finalizzato alla realizzazione di una piattaforma web tematica per la condivisione di informazioni e la vendita diretta di prodotti alla clientela.

	\section{Introduzione}
	L'idea di progetto trae origine e ispirazione dalle constatazioni dirette del referente aziendale circa la crisi endemica dei piccoli e medi produttori vitivinicoli, incapaci di sostenere la concorrenza delle grandi realtà industriali sul piano economico e pubblicitario.

	L'impossibilità di offrire i medesimi prezzi al dettaglio e i maggiori costi di gestione connessi alla ridotta scala produttiva hanno contribuito ad aggravare ulteriormente, in un periodo caratterizzato da una congiuntura economica sfavorevole, la loro condizione.

	Nello sforzo di cercare una soluzione in grado di risollevarne le sorti, riuscendo a valorizzare la superiore qualità dei prodotti e incrementando al contempo il bacino di clientela, è stato individuato nel rapporto diretto tra produttori e consumatori un elemento chiave, capace di favorirne e sostenerne la ripresa sensibilizzando la clientela (attuale e potenziale) sulla qualità della produzione.

	D'altro canto la formula della vendita diretta di prodotti agroalimentari ha riscosso un notevole successo negli ultimi anni assumendo forme e connotazioni differenti, come la filosofica dei consumi a \underline{chilometro zero} e i \underline{gruppi d'acquisto}, che hanno finito con l'essere ripresi e confluire nell'idea di progetto pur in una visione e concezione più ampie.

	La scelta di realizzare una piattaforma web tematica, che sia in grado di raccogliere un vasto numero di utenti interessati alla specifica tipologia di prodotto, ha rappresentato la risposta al secondo (ma non secondario) obiettivo, ossia l'esigenza di conseguire maggiore visibilità presso la potenziale clientela (locale e nazionale, in primis).

	\section{Reti sociali}
	Una piattaforma tematica simile a quella descritta tende inevitabilmente e spontaneamente a raccogliere consensi tra gli utenti che condividano l'interesse specifico.

	\begin{itemize}
	\item potenzialità al di fuori dell'ambito iniziale
	\item multidisciplinarietà: coinvolgimento di figure professionali provenienti da numerosi e variegati settori professionali (psicologia, sociologia, marketing, economia, informatica, ingegneria, \ldots)
	\end{itemize}

	\section{Social (Life) Shuttle}

	%\part{Stage}	

	% CHAPTER
	\chapter{Stage}

	%\section{Criterio di classificazione}

	%\subsection{Modello relazionale}

	%\section{Interfaccia grafica}	

	% CHAPTER
	\chapter{Valutazioni finali}
	% \section{Competenze professionali acquisite}

	% APPENDIX
	\appendix
	\chapter[Glossario]{Glossario}
	\section*{C}
	\paragraph{Chilometri zero}
	\section*{G}
	\paragraph{Gruppi di acquisto}

\end{document}
