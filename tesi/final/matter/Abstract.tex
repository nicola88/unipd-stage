%*******************************************************
% Abstract
%*******************************************************
%\renewcommand{\abstractname}{Abstract}
\pdfbookmark[1]{Sommario}{Sommario}
\begingroup
\let\clearpage\relax
\let\cleardoublepage\relax
\let\cleardoublepage\relax

\chapter*{Sommario}
	L'attività di stage si è svolta presso l'azienda \textit{Sintesi Sas}, che opera nel settore ICT (\textit{Information and Comunication Technology}) realizzando software ERP e piattaforme Web per aziende, in particolare attive nel settore turistico, e fornendo servizi di consulenza e di formazione di imprenditori nell'ambito del marketing strategico, operativo e del controllo di gestione.
	
	Il prodotto di punta dell'azienda - \textit{Planet Hotel} - costituisce un sistema software per la gestione alberghiera tra i più flessibili, ampi e completi presenti nel panorama italiano, in grado di coprire la maggior parte delle necessità aziendali: oltre alla gestione delle prenotazioni e dei conti, esso offre un insieme di moduli integrati per supportare il controllo di gestione e degli interventi di marketing.

	Si tratta di una realtà imprenditoriale a clientela nazionale con sede unica a Mestre (VE), la cui direzione e amministrazione è affidata al solo fondatore, che ha assunto il ruolo di tutor esterno e referente aziendale per l'intera durata dello stage.

	Le attività svolte si inseriscono nell'ambito di un progetto esterno rispetto al business dell'azienda, finalizzato alla realizzazione di una piattaforma web tematica per la condivisione di informazioni e la vendita diretta di prodotti alla clientela e affidato ad un team costituito da differenti figure professionali (sociologi, informatici, ingegneri, \ldots).
	
	\section*{Contenuti}
	\label{ch:tesi:intro:contenuti}
	Il presente documento costituisce una relazione dettagliata in merito all'attività di stage svolta dallo studente Nicola Moretto presso l'azienda \textit{Sintesi Sas}. I contenuti sono organizzati nei seguenti capitoli:
	\begin{description}
	  \item[\nameref{ch:tesi:progetto}] \hfill \\
	  Il primo capitolo illustra le strategie dell'azienda e gli obiettivi, i requisiti e i vincoli del progetto in cui si inseriscono le attività di stage.
	  \item[\nameref{ch:tesi:stage}] \hfill \\ 
		Il secondo capitolo illustra gli obiettivi, i requisiti e l'organizzazione (piano e norme di lavoro) delle attività di stage. A seguire vengono presentate le scelte più rilevanti effettuate e i risultati conseguiti.
	  \item[\nameref{ch:tesi:conclusioni}] \hfill \\
		Il terzo capitolo presenta un'analisi critica a posteriori dell'attività di stage: raggiungimento degli obiettivi prefissati, competenze professionali acquisite, \ldots\ .
	\end{description}

	\section*{Convenzioni tipografiche}
	\label{ch:tesi:intro:convenzioni}
	Al fine di agevolare la consultazione del documento sono state adottate alcune convenzioni tipografiche illustrate di seguito.

	\paragraph{Glossario} Gli acronimi, le abbreviazioni, i nomi propri e i termini specialistici contenuti nel presente documento sono illustrati nel \textit{\nameref{ch:tesi:appendice:glossario}}, consultabile in appendice, al fine di agevolare la lettura e la comprensione degli argomenti trattati. La prima occorrenza di ciascun termine o espressione presente nel glossario appare \underline{sottolineata}.

	\paragraph{Terminologia} I termini propri o di provenienza straniera divenuti di uso corrente nella lingua italiana sono evidenziati in \textit{corsivo}, mentre la prima occorrenza di parole o espressioni che assumono particolare significato nel presente contesto è riportata in \textsc{maiuscoletto}.

	\paragraph{Codice e formule} I nomi di tabelle, classi, package, \ldots\ impiegano uno stile di carattere \textsf{sans serif}, mentre i frammenti di codice impiegano un carattere a \texttt{spaziatura fissa}.
\endgroup		

\vfill
