%************************************************
\chapter{Introduzione}
\label{ch:tesi:intro}
%************************************************
\section{Contenuti}
\label{ch:tesi:intro:contenuti}
Il presente documento costituisce una relazione dettagliata in merito all'attività di stage svolta dallo studente Nicola Moretto presso l'azienda \textit{Sintesi Srl}. I contenuti sono organizzati nei seguenti capitoli:
\begin{description}
  \item[\nameref{ch:tesi:intro}] \hfill \\
  Il primo capitolo illustra brevemente la struttura del documento e le convenzioni tipografiche utilizzate.
  \item[\nameref{ch:tesi:progetto}] \hfill \\
  Il secondo capitolo illustra le strategie dell'azienda e gli obiettivi, i requisiti e i vincoli del progetto in cui si inseriscono le attività di stage.
  \item[\nameref{ch:tesi:stage}] \hfill \\ 
	Il terzo capitolo illustra gli obiettivi, i requisiti e l'organizzazione (piano e norme di lavoro) delle attività di stage. A seguire vengono presentate le scelte più rilevanti effettuate e i risultati conseguiti.
  \item[\nameref{ch:tesi:valutazioni}] \hfill \\
	Il quarto capitolo presenta un'analisi critica a posteriori dell'attività di stage: raggiungimento degli obiettivi prefissati, competenze professionali acquisite, \ldots.
\end{description}

\section{Convenzioni tipografiche}
\label{ch:tesi:intro:convenzioni}
Al fine di agevolare la consultazione del documento sono state adottate alcune convenzioni tipografiche illustrate di seguito.

\paragraph{Glossario} Gli acronimi, le abbreviazioni, i nomi propri e i termini specialistici contenuti nel presente documento sono illustrati nel \textit{\nameref{ch:appendice:glossario}}, consultabile in appendice, al fine di agevolare la lettura e la comprensione degli argomenti trattati. La prima occorrenza di ciascun termine o espressione presente nel glossario appare \underline{sottolineata}.

\paragraph{Terminologia} La prima occorrenza di termini propri o di provenienza straniera divenuti di uso corrente nella lingua italiana sono evidenziati in \textit{corsivo}, mentre le parole o espressioni che assumono particolare significato nel presente contesto sono riportate in \textsc{maiuscoletto}.

\paragraph{Codice e formule} I nomi di tabelle, classi, package, \ldots\ impiegano uno stile di carattere \textsf{sans serif}, mentre i frammenti di codice o formule impiegano un carattere a \texttt{spaziatura fissa}.
