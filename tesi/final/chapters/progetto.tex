\chapter{Progetto}
\label{ch:tesi:progetto}

\section{Genesi}
\label{sec:tesi:progetto:genesi}
L'idea della piattaforma \textit{Social (Life) Shuttle} nasce nel 2010 da un progetto concepito per dar vita ad una comunità virtuale destinata agli artisti sconosciuti e accessibile in mobilità mediante un'applicazione dedicata, \textit{ArtYR}.

Nello stesso periodo una consulenza nell'ambito dei sistemi informativi territoriali ad un'azienda di Bolzano conduce allo sviluppo di un'innovativa piattaforma software: un sistema informativo territoriale in cui l'erogazione di informazioni turistiche è integrata con la vendita di servizi collegati.

Il progetto evolve - grazie alla partecipazione di Comuni, Province e Regioni - in una rete tematica di agenzie di viaggio con un'identità comune e finalizzata alla fusione dei sistemi informativi distrettuali e di vendita.

L'architettura di \textit{Social (Life) Shuttle} trae profonda ispirazione, integrando tre componenti differenti:
\begin{description}
	\item[Business] \hfill \\
	Vendita diretta di prodotti alla clientela.
	\item[Sociale] \hfill \\
	Creazione e sviluppo delle relazioni sociali attraverso la condivisione di informazioni e conoscenza.
	\item[Territorio] \hfill \\
	Sistema di erogazione di informazioni turistiche e territoriali.
\end{description}

\section{Reti sociali}
\label{sec:tesi:progetto:reti-sociali}
Il modello sociologico di \underline{rete sociale} non ha attualmente riscontro presso le piattaforme web di condivisione dei contenuti (\textit{blog}, \textit{forum}, \ldots) o i \textit{social network} (\textit{Facebook}, \textit{Twitter}, \ldots), che si limitano a considerarne e concretizzarne singoli aspetti.

Nelle moderne reti sociali è infatti assente l'incentivo alla condivisione e distribuzione della conoscenza, fattore cruciale per l'aggregazione fisica dei membri delle comunità, da intendersi a sua volta come aggregazioni formatesi intorno ed attraverso la manifestazione di interesse nei confronti di uno specifico tema di dialogo o discussione, che attraversa la sfera individuale, intima e personale dei suoi membri.

Il progetto \textit{Social (Life) Shuttle} rappresenta una nuova generazione di piattaforma di socializzazione, in cui il web diventa solamente un canale di condivisione e un serbatoio della conoscenza generata dalla dialettica tra persone e dove vengono integrati i canoni classici di \textit{blog}, \textit{forum}, \textit{social network} e \textit{media}.

Una relazione sociale nata e costruita su un interesse comune stravolge l'attuale paradigma delle reti sociali virtuali, in cui il legame nasce a prescindere dalla presenza di interessi comuni o informazioni da condividere, e favorisce l'incontro tra persone aventi esperienze simili frutto di tali interessi condivisi.

Ove l'esperienza riguardi anche beni o prodotti, la componente \textit{business} intende offrire ai membri la possibilità di interagire con i produttori, anch'essi attori della comunità.

L'architettura di \textit{Social (Life) Shuttle} consente di declinare la piattaforma in innumerevoli varianti, applicabili ai temi più svariati: al momento sono in fase di sperimentazione per il mondo del vino, il cibo biologico, l'arte commercializzabile e l'attività di ricerca e progettazione collaborativa. 

\section{Architettura}
\label{sec:tesi:progetto:architettura}
Tale piattaforma presenta numerose aspetti che la differenziano dalla concorrenza attuale:
\begin{itemize}
	\item profonda integrazione degli aspetti \textit{social} e \textit{business};
	\item nessuna distinzione tra creatori e fruitori dei contenuti (ciascun membro può condividere le proprie esperienze, segnalare eventi, pubblicare articoli critici, \ldots);
	\item l'autorevolezza di ciascun membro della comunità si rafforza o si indebolisce a seconda della qualità dei contenuti pubblicati, dei giudizi degli altri membri e di altri criteri di valutazione;
	\item lo sfruttamento di tecnologie e dispositivi mobili per favorire la crescita di relazioni al di fuori dell'ambito virtuale della piattaforma (partecipazione ad eventi, raccolta e condivisione di informazioni geolocalizzate, \ldots).
\end{itemize}

\begin{figure}[ht]
	\begin{center}
		\includegraphics[width=6cm]{img/gerarchia-utenti.png}
		\label{fig:tesi:progetto:gerarchia-utenti}
		\caption{Gerarchia degli utenti nelle piattaforme web tradizionali}
	\end{center}
\end{figure}

% definizione di rete sociale: Granovetter.
Per quanto concerne le attività di stage, due aspetti della piattaforma assumono particolare rilevanza: i contenuti informativi e i relativi criteri di classificazione.

\subsection{Contenuti informativi}
\label{sec:tesi:progetto:contenuti}
I contenuti informativi rappresentano lo strumento essenziale per la condivisione delle esperienze e della conoscenza intorno al tema specifico della piattaforma.

Per individuare le classi di contenuti adatte a esprimere in una forma strutturata le informazioni si è tratta ispirazione dalle forme espressive e comunicative tipiche della dialettica quotidiana, poiché immediatamente e intuitivamente comprensibili agli utenti, a prescindere dal loro livello di esperienza.

In particolare, si distinguono la natura della comunicazione, connessa allo scopo e al tono con cui ci esprimiamo, e il formato delle informazioni, che dipendono strettamente dai sensi e dai canali di comunicazione a disposizione per scambiare informazioni con l'interlocutore, sia esso un individuo singolo o un gruppo.

\paragraph{Classi}
I tipi di contenuto pubblicabili nella piattaforma dovrebbero essere in numero adeguato a coprire il maggior numero possibile di esigenze comunicative pur rimanendo facilmente e intuitivamente distinguibili, ossia l'utente non dovrebbe nutrire dubbi circa il più adatto a formalizzare di volta in volta l'informazione che desidera condividere.
\begin{description}
\item[Domanda] \hfill \\
La domanda classica rende particolarmente esplicito lo scopo della comunicazione, ossia la richiesta di informazioni di varia natura agli altri utenti della piattaforma. Si distingue in pubblica o privata, a seconda che l'utente desideri rivolgerla ad un particolare sottoinsieme di utenti. 
\item[Risposta] \hfill \\
Duale della domanda, la risposta è anch'essa in forma pubblica o privata per consentire all'utente di renderla accessibile e consultabile solo a certi utenti, spesso l'autore della domanda a cui risponde.
\item[Pensierino] \hfill \\
Il pensierino rappresenta una forma di comunicazione adatta ad esprimere un contenuto di lunghezza breve e prevalentemente superficiale (considerazioni, stati d'animo, freddure, \ldots).
\item[Evento] \hfill \\
L'evento aiuta a promuovere qualsiasi iniziativa che rientri nell'ambito tematico della piattaforma e cui possano prender parte altre persone (incontro pubblico, concerto, fiera, \ldots).
\item[Discorso] \hfill \\
Il discorso identifica un contenuto articolato, sia nella forma sia nei contenuti, destinato alla condivisione di informazioni dettagliate e approfondite.
\item[Recensione] \hfill \\
La recensione esprime un giudizio critico nei confronti di un prodotto specifico. 
\item[Comunicazione privata] \hfill \\
La comunicazione privata è l'unica forma di contatto diretto e riservato tra due utenti.
\end{description}

\paragraph{Elementi}
Ciascun tipo di contenuto esprime un intento comunicativo ben preciso, ma non è vincolato ad una struttura e ad un formato predefiniti: la classe, che esprime l'intento della comunicazione, si colloca in un piano distinto rispetto al formato, ossia la struttura e le caratteristiche specifiche del contenuto informativo condiviso.

Ove la dialettica quotidiana dispone infatti di cinque sensi e può esprimersi in forma non solo verbale, nel web gli utenti sperimentano differenti forme di comunicazione: contenuti testuali e grafici, flussi audio e video, documenti elettronici, messaggistica istantanea, \ldots\ .

I contenuti informativi non presentano dunque una struttura fissa a seconda della classe, ma possono essere liberamente redatti a partire da una serie di elementi predefiniti, frutto di una ricerca tra le principali e più diffuse piattaforme web disponibili (\textit{blog}, \textit{forum}, \textit{social network}, \textit{chat}, \ldots) e di una successiva analisi e rielaborazione dei risultati ottenuti:
\begin{description}
\item[Audio] \hfill \\
Contenuto audio statico o in tempo reale (\textit{live streaming}, \ldots).
\item[Immagini] \hfill \\
Contenuto grafico statico.
\item[Video] \hfill \\
Contenuto video statico o in tempo reale (\textit{live streaming}, \ldots).
\item[Sondaggio] \hfill \\
Domanda a risposta multipla.
\item[Documento] \hfill \\
File di testo o binario caricato nella piattaforma.
\item[Stringa] \hfill \\
Contenuto testuale avanzato (intestazioni, formattazione dei caratteri, collegamenti ipertestuali, \ldots).
\item[Citazione] \hfill \\
Citazioni o riferimenti ad altri elementi di un contenuto, ad un contenuto informativo o a prodotti presenti nella piattaforma. 
\end{description}

\begin{figure}[ht]
	\begin{center}
		\includegraphics{img/placeholder.png}
		\label{fig:tesi:progetto:contenuti-informativi}
		\caption{Struttura di un contenuto informativo}
	\end{center}
\end{figure}

La struttura modulare dei contenuti informativi consente di riusare, riferire o citare gli elementi costituenti e di catalogarli con maggior facilità e precisione, riuscendo a classificare ciascun frammento di informazione presente al loro interno.

\paragraph{Proprietà}
I contenuti informativi - a prescindere dalla classe e dalla struttura - presentano un insieme di proprietà comuni, alcune delle quali assumono particolare rilevanza per l'attività di stage, rappresentando utili criteri addizionali per filtrare i contenuti informativi durante una ricerca: si tratta di \textsc{autore}, \textsc{data di pubblicazione} e \textsc{tipo} del contenuto.

\paragraph{Relazioni}
Ove tradizionalmente ci si affida ai commenti per consentire agli utenti di esprimere un'opinione rispetto alle informazioni riportate o alle posizioni espresse in un contenuto, in \textit{Social (Life) Shuttle} si permette di rispondere ad un contenuto pubblicato nella piattaforma direttamente con altri contenuti, in numero arbitrario.

La relazione di dipendenza tra i contenuti prescinde dalla classe specifica, non ponendo vincoli di alcun genere circa la classe ed il formato della risposta ad un contenuto informativo.

Ciò consente maggiore libertà all'utente nello scegliere la forma espressiva più adeguata per condividere il proprio messaggio, ne facilita la catalogazione e allo stesso tempo rispecchia il principio di uguaglianza tra gli utenti espresso in precedenza ed elemento cardine della piattaforma.

Nel corso del tempo a partire da ciascun contenuto informativo possono così svilupparsi e ramificarsi diverse \textsc{discussioni}, senza limiti di ampiezza o profondità.

\subsection{Sistema di classificazione}
\label{sec:tesi:progetto:classificazione}
Il sistema di classificazione consiste in un insieme di criteri che associano a ciascun contenuto alcuni metadati, in grado di fornire agli utenti della piattaforma informazioni utili a contestualizzarlo, ad interpretarlo e a valutarne l'interesse soggettivo:
\begin{description}
 	\item[Argomento] \hfill \\
 	L'argomento di un contenuto rappresenta la branca del sapere - agnostica rispetto al tema specifico della piattaforma - cui appartiene. 
 	\item[Emozioni] \hfill \\
 	Le emozioni indicano lo stato d'animo con cui un contenuto sia stato pubblicato dall'autore.
 	\item[Giudizi] \hfill \\
 	I giudizi forniscono una valutazione qualitativa sul contenuto e sono espressi dagli utenti.
 	\item[Intenzioni] \hfill \\
 	Le intenzioni indicano lo spirito con cui l'autore redige il contenuto (opinione, critica, \ldots).
 	\item[Interessi] \hfill \\
 	Gli interessi rappresentano temi specifici della piattaforma nei confronti dei quali ciascun utente registrato dichiara di nutrire passione.
\end{description}

Queste meta-informazioni assumono particolare rilevanza nel processo di ricerca di informazioni all'interno della piattaforma, poiché consentono di escludere o meno determinati contenuti dai risultati.

Tra quelli evidenziati, tuttavia, spicca l'assenza di un criterio in grado di catalogare ordinatamente le informazioni presenti nei contenuti per facilitarne la ricerca, il reperimento e la consultazione: il primo obiettivo dell'attività di stage consiste nell'individuare un meccanismo efficiente per rendere più agevole la consultazione della conoscenza custodita nella piattaforma, fornendo un livello di astrazione rispetto alla semplice enumerazione dei contenuti.

\section{Requisiti e vincoli}
\label{sec:tesi:progetto:requisiti}
Durante gli incontri preliminari all'attività di stage sono stati fissati gli obiettivi, i requisiti ed i vincoli concernenti le attività previste ed i prodotti attesi. 

\subsection{Criterio di classificazione}
\label{sec:tesi:progetto:requisiti:criterio-classificazione}
La progettazione del criterio di classificazione deve tenere conto di alcuni vincoli e requisiti riguardanti l'architettura della piattaforma e l'integrazione con il sistema di classificazione:

\begin{description}
	\item[Indipendenza dai criteri esistenti] \hfill \\
	Il criterio deve minimizzare il grado di accoppiamento per risultare facilmente mantenibile e aggiornabile senza intaccare lo stato, l'integrità e le funzionalità dei rimanenti e deve tenere conto di possibili evoluzioni della piattaforma, che comportino l'aggiornamento o la rimozione dei criteri esistenti o l'aggiunta di nuovi.
	\item[Indipendenza dalle classi di contenuti] \hfill \\
	Il criterio non deve distinguere tra contenuti di classi diverse, ma deve considerare esclusivamente le proprietà e le relazioni definite o definibili sul contenuto generico.
	\item[Indipendenza dalle componenti del sistema] \hfill \\
	Il criterio deve minimizzare le dipendenze e l'accoppiamento con le altre componenti del sistema, che possono essere soggette ad aggiornamenti sostanziali (in particolare quelle di terze parti) o interventi di manutenzione evolutiva.
	\item[Modularità] \hfill \\
	Il criterio dev'essere progettato in modo tale da potersi avvantaggiare - in futuro - di soluzioni tecniche o tecnologiche in grado di automatizzare (in parte o del tutto) le operazioni di classificazione dei contenuti.
\end{description}

Le potenziali criticità tecniche, legate all'implementazione del criterio di classificazione, devono essere raccolte e condivise con il team di progetto, che provvederà a valutarle, a fornire eventuali indicazioni e ad individuare le soluzioni ritenute appropriate e compatibili con l'architettura e le specifiche della piattaforma.

\subsection{Interfaccia grafica}
\label{sec:tesi:progetto:requisiti:interfaccia-grafica}
L'interfaccia grafica per la consultazione dei risultati di una ricerca sui contenuti informativi deve soddisfare alcuni requisiti essenziali:

\begin{itemize}
	\item deve consentire all'utente di inserire dei termini di ricerca e selezionare un ambito;
	\item deve potersi interfacciare a componenti terze per ottenere i risultati di ricerca;
	\item deve mostrare i contenuti con forme geometriche elementari, che permettano di distinguerne intuitivamente la classe di appartenenza;
	\item deve permettere la consultazione della \textsc{discussione} associata ad un contenuto, evidenziando il flusso informativo (le sequenze di risposte ad un contenuto);
	\item dovrebbe gestire dei filtri basati sugli interessi ed il livello di esperienza di un utente registrato;
	\item dev'essere in grado di visualizzare ordinatamente un numero elevato di risultati di ricerca, evitando un sovraccarico cognitivo;
	\item dev'essere utilizzabile agevolmente da utenti con differenti livelli di esperienza;
	\item dev'essere adeguatamente fruibile su dispositivi mobili.
\end{itemize}
