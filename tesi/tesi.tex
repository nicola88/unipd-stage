% **************************************************************************************************************
% A Classic Thesis Style
% An Homage to The Elements of Typographic Style
%
% Copyright (C) 2011 Andr\'e Miede http://www.miede.de
%
% If you like the style then I would appreciate a postcard. My address 
% can be found in the file ClassicThesis.pdf. A collection of the 
% postcards I received so far is available online at 
% http://postcards.miede.de
%
% License:
% This program is free software; you can redistribute it and/or modify
% it under the terms of the GNU General Public License as published by
% the Free Software Foundation; either version 2 of the License, or
% (at your option) any later version.
%
% This program is distributed in the hope that it will be useful,
% but WITHOUT ANY WARRANTY; without even the implied warranty of
% MERCHANTABILITY or FITNESS FOR A PARTICULAR PURPOSE.  See the
% GNU General Public License for more details.
%
% You should have received a copy of the GNU General Public License
% along with this program; see the file COPYING.  If not, write to
% the Free Software Foundation, Inc., 59 Temple Place - Suite 330,
% Boston, MA 02111-1307, USA.
%
% **************************************************************************************************************
% Note:
%    * You must not use "u etc. in strings/commands that will be spaced out (use \"u or real umlauts instead)
%    * New enumeration (small caps): \begin{aenumerate} \end{aenumerate}
%    * For margin notes: \marginpar or \graffito{}
%    * Do not use bold fonts in this style, it is designed around them
%    * Use tables as in the examples
%    * See classicthesis-preamble.sty for useful commands
% **************************************************************************************************************
% To Do:
%    * Eliminare l'opzione draft e drafting dal comando \documentclass
%    * [Opzionale] Aggiungere le opzioni twoside + BCOR=5mm e "decommentare" i margini nella pagina di titolo
% **************************************************************************************************************
\documentclass[a4paper,openright,numbers=noenddot,headinclude,footinclude,abstracton,11pt]{scrreprt}
\usepackage[italian]{babel}
\usepackage{makeidx}
\usepackage{parskip}
\usepackage{textcomp}
\usepackage[utf8]{inputenc}
\input{tesi-config}
\hyphenation{}
\makeindex

\begin{document}
\pagenumbering{arabic}
\pagestyle{scrheadings} %alternativa: headings%
\frenchspacing
\raggedbottom
%\renewcommand*{\bibname}{new name}
%\setbibpreamble{}

%********************************************************************
% Frontmatter
%*******************************************************
\include{matter/Titlepage}
\thispagestyle{empty}

\hfill

\vfill

\noindent \textit{\myTitle} \\
\mySubtitle \\
\textcopyright\ \myTime

\medskip

\noindent
\spacedlowsmallcaps{Studente}: \\
\myName (matr. 578258)

\medskip

\noindent\spacedlowsmallcaps{Relatore}: \\
\mySupervisor

\medskip

\noindent\spacedlowsmallcaps{Luogo}: \\
\myLocation

\medskip

\noindent\spacedlowsmallcaps{Data}: \\
\myTime

%*******************************************************
% Dedication
%*******************************************************
\thispagestyle{empty}
%\phantomsection
\refstepcounter{dummy}
\pdfbookmark[1]{Dedica}{Dedication}

\vspace*{3cm}

\begin{flushright}
	\emph{Ad Alessandro, Christian, Ester, Marta e Romeo con cui tutto ha avuto inizio.}
	
	\emph{Al dott. Bovo per la passione con cui mi ha seguito.}
	
	\emph{Al prof. Vardanega per la disponibilità con cui mi ha seguito.}
	
	\emph{Alla mia famiglia che mi ha accompagnato sin qui.}
\end{flushright}

%*******************************************************
% Abstract
%*******************************************************
%\renewcommand{\abstractname}{Abstract}
\pdfbookmark[1]{Abstract}{Abstract}
\begingroup
\let\clearpage\relax
\let\cleardoublepage\relax
\let\cleardoublepage\relax

\chapter*{Sommario}
Breve presentazione dei contenuti del documento\dots

\endgroup			

\vfill
\include{matter/Acknowledgments}
\include{matter/Contents}

%********************************************************************
% Parte 1
%*******************************************************
% use \cleardoublepage here to avoid problems with pdfbookmark
\ctparttext{Presentazione di \emph{Social (Life) Shuttle}.}
\part{La piattaforma}
%************************************************
\chapter{Classi}\label{ch:contenuti:classi}
%************************************************
\section{Documento}
\section{Domanda}
\section{Evento}
\section{Multimedia}
\section{Pensiero}
\section{Risposta}
%*****************************************


%********************************************************************
% Parte 2
%*******************************************************
\ctparttext{Panoramica e presentazione delle classi di contenuti informativi della piattaforma.}
\part{Contenuti informativi}
%*****************************************
\chapter{Panoramica}\label{ch:panoramica}
%*****************************************
Scrivere qui...
%*****************************************

%************************************************
\chapter{Introduzione}\label{ch:classificazione:intro}
%************************************************
Introduzione ai sistemi di classificazione esistenti e ai relativi pro e contro.
%*****************************************

%************************************************
\chapter{Eventi}\label{ch:eventi}
%************************************************
Scrivere qui...
%*****************************************


%********************************************************************
% Parte 3
%*******************************************************
\ctparttext{Panoramica e presentazione dei sistemi di classificazione utilizzati nella piattaforma.}
\part{Sistemi di classificazione}
%************************************************
\chapter{Progettazione}\label{ch:classificazione:design}
%************************************************
Progettazione del sistema di classificazione.
%*****************************************

%************************************************
\chapter{Etichette}\label{ch:eventi}
%************************************************
Dizionario
%*****************************************


%********************************************************************
% Parte 3
%*******************************************************
\ctparttext{Panoramica e presentazione dell'interfaccia di visualizzazione dei risultati di ricerca e delle relazioni tra i contenuti informativi.}
\part{Ricerca e navigazione}

% ********************************************************************
% Backmatter
%*******************************************************
\appendix
\part{Appendice}
%********************************************************************
% Appendix
%*******************************************************
% If problems with the headers: get headings in appendix etc. right
%\markboth{\spacedlowsmallcaps{Appendix}}{\spacedlowsmallcaps{Appendix}}
\chapter{Glossario}
% Termini sottolineati in rosso nel quaderno degli appunti
\section*{A}
\paragraph*{Accezione}
\section*{B}
\section*{C}
\paragraph*{Classificatore}
\section*{D}
\section*{E}
\paragraph*{Etichetta}
\section*{F}
\section*{G}
\section*{H}
\section*{I}
\section*{J}
\section*{K}
\section*{L}
\section*{M}
\section*{N}
\section*{O}
\section*{P}
\section*{Q}
\section*{R}
\section*{S}
\paragraph*{Sinonimo}
\section*{T}
\section*{U}
\section*{V}
\section*{W}
\section*{X}
\section*{Y}
\section*{Z}

\cleardoublepage\include{matter/Bibliography}
%\cleardoublepage\printindex
\end{document}
