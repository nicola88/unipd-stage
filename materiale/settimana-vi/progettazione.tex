\documentclass[10pt,a4paper,headinclude,footinclude,hidelinks]{scrreprt} % KOMA-Script
\usepackage[italian]{babel}
\usepackage[utf8]{inputenc}
\usepackage[T1]{fontenc}
\usepackage{graphicx}
\usepackage{amsfonts}
\usepackage[]{../../classicthesis} % nochapters
\pagestyle{scrheadings}
\setcounter{tocdepth}{2}

\begin{document}
    \title{\rmfamily\normalfont\spacedallcaps{Progettazione}}
    \author{\spacedlowsmallcaps{Nicola Moretto (matr. 578258)}}
    \date{\today}
    
    \maketitle
    
    \begin{abstract}
        \noindent Il documento riporta le informazioni di progettazione riguardanti l'interfaccia grafica per la visualizzazione e la navigazione dei contenuti.
    \end{abstract}
    
	\begin{table}[ht]
	\centering
	\begin{tabular}{|c|c|l|}
	\hline
	\textsc{Versione} & \textsc{Data} & \textsc{Modifiche} \\ \hline
	0.1 & 15-10-2012 & Stesura iniziale del documento. \\ \hline
	0.2 & 17-10-2012 & Redatto il capitolo \ref{ch:stage:design:architettura}. \\ \hline
	0.3 & 18-10-2012 & Redatte le sezioni \ref{sec:stage:design:sistema:model.filter}, \ref{sec:stage:design:sistema:view.filter} e \ref{sec:stage:design:sistema:view.search}. \\ \hline
	0.4 & 19-10-2012 & Redatti i capitoli \ref{ch:stage:design:model} e \ref{ch:stage:design:view}. \\ \hline
	\end{tabular}
	\caption{Registro delle modifiche}
	\label{tab:stage:wp:workload}
	\end{table}

	\tableofcontents

	%----------
	% CAPITOLO
	%----------
	\chapter{Introduzione}
	\label{ch:stage:design:intro}

	\section{Obiettivi}
	\begin{itemize}
	\item aggiunta, rimozione o aggiornamento (motore di ricerca) di nuovi ambiti di ricerca;
	\end{itemize}

	\section{To do}
	\begin{itemize}
	\item Aggiungere la lista di informazioni essenziali e aggiuntive di un contenuto nei requisiti.
	\end{itemize}

	\section{Riferimenti informativi}
	\begin{enumerate}
	\item Analisi dei requisiti (\textit{analisi\_dei\_requisiti\_1.0} allegata alla presente documentazione);
	\item Sistema di classificazione (\textit{sistema\_di\_classificazione\_2.0} allegato alla presente documentazione).
	\end{enumerate}

	%----------
	% CAPITOLO
	%----------
	\chapter{Architettura}
	\label{ch:stage:design:architettura}

	\section{Architettura generale}
	\label{sec:stage:design:architettura:mvc}
	L'architettura del sistema software rispecchia il design pattern architetturale MVC, che prevede e garantisce la separazione delle tre componenti fondamentali del sistema:
	\begin{description}
	\item[Model] Racchiude i dati e le informazioni dell'applicazione e definisce le modalità di accesso e fruizione degli stessi da parte delle altre componenti (\textit{Controller} e \textit{View}).
 	\item[View] Rappresenta l'interfaccia grafica mediante la quale vengono visualizzate le informazioni e i dati conservati nel \textit{Model} e l'utente può interagire con il sistema. La rilevazione dell'avvenuta interazione dell'utente è responsabilità di tale componente, mentre la gestione della reazione è demandata al \textit{Controller}.
	\item[Controller] Incorpora la logica di controllo dell'applicazione, inizializzando il sistema e traducendo l'interazione dell'utente con l'interfaccia grafica (\textit{View}) in operazioni sui dati (\textit{Model}).
	\end{description}

	\section{Componenti architetturali}
	\label{sec:stage:design:mvc}

	\subsection{Componente Model}
	\label{sec:stage:design:mvc:model}
	La componente \textit{model} conserva tutte i tipi di informazioni connessi alla ricerca:
	\begin{itemize}
	\item gli ambiti (etichette, frasi);
	\item i criteri (termini di ricerca, entità);
	\item i filtri (argomento, data di pubblicazione, emozioni, \ldots);
	\item i risultati (contenuti informativi).
	\end{itemize}

	\subsection{Componente View}
	\label{sec:stage:design:mvc:view}
	La componente \textit{view} rappresenta l'interfaccia grafica mediante la quale l'utente interagisce con il sistema per effettuare una ricerca, raffinarne i criteri o consultarne i risultati.

	\paragraph{Livelli}
	Essa è organizzata in quattro livelli (o strati) distinti, che includono rispettivamente:
	\begin{itemize}
	\item gli strumenti e le informazioni connesse alla ricerca (barra di ricerca, etichette, entità, \ldots);
	\item i filtri di ricerca;
	\item i risultati della ricerca (proprietà e relazioni dei contenuti informativi;
	\item la cronologia (organizzazione temporale dei contenuti).
	\end{itemize}

	\subsection{Componente Controller}
	\label{sec:stage:design:mvc:controller}
	La componente \textit{controller} gestisce l'interazione dell'utente con l'interfaccia grafica e le operazioni connesse alla ricerca e alla visualizzazione dei contenuti informativi, tra cui:
	\begin{itemize}
	\item la configurazione dei criteri di ricerca (selezione di un'entità di un'etichetta, gestione delle entità);
	\item il reperimento dei contenuti informativi corrispondenti ai criteri di ricerca;
	\item la creazione e la gestione dei filtri di ricerca;
	\item l'aggiornamento dei risultati di ricerca visualizzati a fronte di modifiche alle entità e ai filtri di ricerca;
	\item la consultazione dei contenuti informativi da parte dell'utente;
	\item \ldots
	\end{itemize}

	%----------
	% CAPITOLO
	%----------
	\chapter{Componente Model}
	\label{ch:stage:design:model}
	Questo capitolo illustra i componenti del \textit{Model}, per ciascuno dei quali è indicato il nome della classe accompagnata da un'identificazione sintetica, separati dal carattere '|' (separatore verticale).

	% SECTION
	\section{Package model}
	\label{sec:stage:design:sistema:model}

	\subsection[Content]{Content | Contenuto informativo}
	\label{sec:stage:design:sistema:model:content}
	La classe rappresenta un generico contenuto informativo pubblicato dagli utenti nella piattaforma.

	\paragraph{Propriet\`a}
	\begin{itemize}
	\item Argomento
	\item Autore
	\item Data di pubblicazione
	\item Giudizio
	\item Tipo
	\item Titolo
	\end{itemize}

	\subsection[Entity]{Entity | Entità del dominio}
	\label{sec:stage:design:sistema:model:entity}
	La classe modella le entità del dominio della piattaforma.

	\subsection[Label]{Label | Etichetta del dizionario}
	\label{sec:stage:design:sistema:model.search:label}
	La classe modella le etichette del dizionario della piattaforma, ciascuna delle quali può possedere molteplici accezioni, cui sono associate altrettante entità distinte.  

	\subsection[Meaning]{Meaning | Accezione di un'etichetta}
	\label{sec:stage:design:sistema:model:meaning}
	La classe modella le entità del dominio della piattaforma.

	\subsection[User]{User | Utente}
	\label{sec:stage:design:sistema:model:user}
	La classe rappresenta un generico utente della piattaforma.

	% SECTION
	\section{Package model.criteria}
	\label{sec:stage:design:sistema:model.criteria}
	Ciascun contenuto può essere classificato in accordo a differenti criteri, ciascuno dei quali prende in esame una differente proprietà del contenuto medesimo, come l'autore, la data di pubblicazione, il tipo, le emozioni associate, \ldots .

	Ciascuna proprietà può essere propria del contenuto o funzionale al contesto di ricerca (attinenza, \ldots).

	\subsection[Criterion]{Criterion | Criterio di classificazione}
	\label{sec:stage:design:sistema:model.criteria:criteria}
	Interfaccia di tutti i criteri di classificazione.

	\subsection[Author]{Author | Autore}
	\label{sec:stage:design:sistema:model.criteria:author}

	\subsection[Emotion]{Emotion | Emozione}
	\label{sec:stage:design:sistema:model.criteria:emotion}

	\subsection[Intention]{Intention | Intenzione}
	\label{sec:stage:design:sistema:model.criteria:intention}

	\subsection[PublicationDate]{PublicationDate | Data di pubblicazione}
	\label{sec:stage:design:sistema:model.criteria:publication-date}

	\subsection[Rating]{Rating | Giudizio}
	\label{sec:stage:design:sistema:model.criteria:rating}

	\subsection[Topic]{Topic | Argomento}
	\label{sec:stage:design:sistema:model.criteria:topic}

	\subsection[Type]{Type | Tipo di contenuto}
	\label{sec:stage:design:sistema:model.criteria:type}

	% SECTION	
	\section{Package model.filter}
	\label{sec:stage:design:sistema:model.filter}
	Ciascun filtro è associato ad una proprietà dei risultati di ricerca (\nameref{sec:stage:design:sistema:model.criteria:criteria}) e partiziona automaticamente l'insieme dei possibili valori in due sottoinsiemi: ammessi o bloccati. Nella configurazione iniziale, tutti i valori possibili di una proprietà sono ammessi, mentre l'utente può intervenire secondo modalità differenti per alterare tale partizionamento (vedi sezione \ref{sec:stage:design:sistema:view.filter}).
	
	Il valore che ciascun contenuto (da intendersi come risultato di una ricerca) ha rispetto ad una proprietà associata ad un filtro può quindi appartenere ad uno dei due sottoinsiemi: il contenuto viene mostrato solo se tutti i valori delle proprietà in questione risultano ammessi.

	\subsection[Filter]{Filter | Filtro di ricerca}
	\label{sec:stage:design:sistema:model.filter:filter}
	Tale componente rappresenta l'interfaccia dei filtri per il raffinamento dei risultati di una ricerca e viene implementata dalle classi che modellano le tipologie standard di filtri di ricerca (\textit{\nameref{sec:stage:design:sistema:model.filter:list-filter}}, \textit{\nameref{sec:stage:design:sistema:model.filter:range-filter}}, \textit{\nameref{sec:stage:design:sistema:model.filter:switch-filter}} e \textit{\nameref{sec:stage:design:sistema:model.filter:value-filter}}).
	
	\subsection[FilterManager]{FilterManager | Gestione dei filtri}
	\label{sec:stage:design:sistema:model.filter:filter-manager}
	Tale componente rappresenta l'interfaccia del package \textit{model.filter}, utile a esporre le funzionalità per l'istanziazione e la gestione dei filtri.
	\paragraph{Design Pattern:} Facade, Singleton

	\subsection[ListFilter]{ListFilter | Filtro con lista di valori}
	\label{sec:stage:design:sistema:model.filter:list-filter}
	Il componente modella un filtro basato su una lista di possibili valori, ciascuno dei quali può essere autorizzato o bloccato dall'utente.

	\subsection[RangeFilter]{RangeFilter | Filtro con intervallo di valori}
	\label{sec:stage:design:sistema:model.filter:range-filter}
	Il componente modella un filtro basato su un intervallo di valori ordinati (numeri, date, \ldots). Siano $inf$ e $sup$ rispettivamente l'estremo inferiore e superiore dell'intervallo: risultano dunque ammessi tutti e soli i valori $v$ tali che $v \in \left[inf,sup\right]$.

	All'utente è consentito scegliere i valori desiderati di $inf$ e $sup$ alle seguenti condizioni:
	\begin{itemize}
	\item $sup$ sia un valore valido o $+\infty$;
	\item $inf$ sia un valore valido o $-\infty$;
	\item $min \leq max$;
	\item se è definito un valore attuale $current$, allora deve valere $min \leq current \leq max$.
	\end{itemize}

	Se l'insieme dei valori ordinati prevede un minimo $min$ e/o un massimo $max$ si aggiungono le seguenti condizioni:
	\begin{itemize}
	\item $sup \leq max$;
	\item $inf \geq min$.
	\end{itemize}

	\subsection[SwitchFilter]{SwitchFilter | Filtro a doppio stato}
	\label{sec:stage:design:sistema:model.filter:switch-filter}
	Il componente modella un filtro il cui partizionamento è predefinito e invariabile e che l'utente può solamente decidere di abilitare o disattivare.

	\subsection[ValueFilter]{ValueFilter | Filtro con soglia di valore}
	\label{sec:stage:design:sistema:model.filter:value-filter}
	Il componente modella un filtro basato su una soglia di valore. Sia $value$ il valore scelto: in tal caso risultano ammessi tutti e soli i valori $x$ validi tali che $x \geq value$; se l'insieme prevede un minimo $min$ e/o un massimo $max$ dev'essere soddisfatta anche la condizione $min \leq x \leq max$.

	% SECTION	
	\section{Package model.search}
	\label{sec:stage:design:sistema:model.search}

	\subsection[SearchEntityList]{SearchEntityList | Gestione della ricerca}
	\label{sec:stage:design:sistema:model.search:search-entity-list}

	\subsection[SearchManager]{SearchManager | Gestione della ricerca}
	\label{sec:stage:design:sistema:model.search:search-manager}

	\paragraph{Design Pattern} Facade

	\subsection[SearchQuery]{SearchQuery | Query di ricerca}
	\label{sec:stage:design:sistema:model.search:search-query}
	Tale componente modella la query di ricerca, ossia la stringa inserita dall'utente nella barra di ricerca (\textit{\nameref{sec:stage:design:sistema:view.search:search-bar}}), contenente una lista di termini o espressioni separati da virgola.

	La classe fornisce i metodi per effettuare l'analisi sintattica della stringa al fine di estrapolare i termini (\textit{\nameref{sec:stage:design:sistema:model.search:search-term}}) da cercare e stabilire quali di essi siano etichette o frasi.

	\subsection[SearchScope]{SearchScope | Ambito di ricerca}
	\label{sec:stage:design:sistema:model.search:search-scope}
	La classe rappresenta il generico ambito di ricerca, che l'utente può selezionare mediante il corrispondente selettore (\textit{\nameref{sec:stage:design:sistema:view.search:search-scope-selector}}).

	\subsection[SearchTerm]{SearchTerm | Chiave di ricerca}
	\label{sec:stage:design:sistema:model.search:search-term}
	La componente rappresenta un'interfaccia comune a tutti i tipi di termini di ricerca, ossia frasi e etichette.

	Ciascun termine di ricerca viene memorizzato all'interno di un'istanza di \textit{\nameref{sec:stage:design:sistema:model.search:search-query}} e viene passato al \textit{SearchManager} per consentire l'esecuzione della ricerca.

	% SECTION	
	\section{Package model.search.provider}
	\label{sec:stage:design:sistema:model.search.provider}

	%----------
	% CAPITOLO
	%----------
	\chapter{Componente View}
	\label{ch:stage:design:view}
	Questo capitolo illustra i componenti del \textit{View}, per ciascuno dei quali è indicato il nome della classe accompagnata da un'identificazione sintetica, separati dal carattere '|' (separatore verticale).

	% SECTION
	\section{Package view}
	\label{sec:stage:design:sistema:view}
	
	\subsection[MainWindow]{MainWindow | Finestra principale}
	La finestra principale rappresenta il contenitore all'interno del quale vengono inseriti e opportunamente collocati tutti gli elementi grafici dell'interfaccia, tra cui gli strumenti di ricerca, i filtri e i contenuti.

	% SECTION
	\section{Package view.content}
	\label{sec:stage:design:sistema:view.content}

	\subsection[ContentView]{ContentView | Contenuto informativo}
	Si tratta del componente grafico deputato a rappresentare graficamente un contenuto informativo, risultato di una ricerca effettuata dall'utente, e le relative informazioni, in formato grafico o testuale e distinte in essenziali o aggiuntive.

	\paragraph{Informazioni essenziali} Le informazioni essenziali sono utili per una immediato e preciso inquadramento del contenuto informativo da parte dell'utente al fine di stabilirne la rilevanza soggettiva.
	\begin{description}
	\item[Autore] L'autore del contenuto è l'utente (\textit{\nameref{sec:stage:design:sistema:model:user}}) che lo ha pubblicato all'interno della piattaforma e viene rappresentato testualmente mediante il suo \textit{nome utente o proprio}.
	\item[Attinenza] Il grado di attinenza di un contenuto rispetto ai criteri di ricerca corrisponde - in percentuale - al rapporto tra le entità assegnate e quelle cercate. Tale informazione viene rappresentata graficamente variando proporzionalmente la \textit{dimensione} dell'elemento grafico.
	\item[Data di pubblicazione] La data di pubblicazione del contenuto viene indicata testualmente.
	\item[Tipo] Il tipo di contenuto viene rappresentato graficamente mediante la forma\footnote{Le forme utilizzate sono elementari per garantire l'immediata riconoscibilità da parte di qualsiasi tipo di utente, evitando l'impiego di forme potenzialmente ambigue o ignote a seconda del suo profilo sociale, culturale, geografico, \ldots} specifica dell'elemento grafico e può essere:
	\begin{itemize}
	\item Comunicazione
	\item Discorso
	\item Domanda
	\item Evento
	\item Pensiero
	\item Recensione
	\item Risposta
	\end{itemize}	 
	\item[Titolo] Il titolo assegnato al contenuto rappresenta un'informazione chiave e viene rappresentata in formato testuale.
	\end{description}

	\paragraph{informazioni aggiuntive} Le informazioni aggiuntive forniscono all'utente dettagli utili per approfondire l'esame di un contenuto.
	\begin{description}
	\item[Argomento] Ciascun contenuto può riferirsi al più ad un argomento, rappresentabile in maniera grafica (colore o simbolo) o testuale.
	\item[Emozioni] A ciascun contenuto possono essere associate diverse emozioni, che esprimono lo stato d'animo dell'autore al momento della sua redazione o pubblicazione. La lista delle emozioni associate ad un contenuto viene visualizzata in formato testuale.
	\item[Entit\`a] A ciascun contenuto possono essere assegnate delle etichette, che riferiscono altrettante entità del dominio da visualizzare in formato testuale (mediante il rispettivo nome).
	\item[Intenzioni] A ciascun contenuto possono essere associate delle intenzioni, che possano fornire delle linee guida interpretative ai lettori. La lista delle intenzioni associate ad un contenuto viene visualizzata in formato testuale.
	\end{description}

	% SECTION
	\section{Package view.filter}
	\label{sec:stage:design:sistema:view.filter}
	Ciascun filtro è associato ad una proprietà dei risultati di ricerca, ossia i contenuti informativi, e partiziona automaticamente l'insieme dei possibili valori in due sottoinsiemi: ammessi o bloccati. Nella configurazione iniziale, tutti i valori possibili di una proprietà sono ammessi, mentre l'utente può intervenire secondo modalità differenti per alterare tale partizionamento.
	
	Il valore che ciascun contenuto (da intendersi come risultato di una ricerca) ha rispetto ad una proprietà associata ad un filtro può appartenere ad uno dei due sottoinsiemi: un contenuto viene mostrato solo se tutti i valori delle proprietà in questione risultano ammessi.

	\subsection[FilterContainer]{FilterContainer | Livello filtri}
	Il \textit{FilterContainer} raccoglie e mantiene separati in un livello distinto gli elementi grafici di tipo \textit{\nameref{sec:stage:design:sistema:view.filter:filter}}, che forniscono all'utente gli strumenti per regolare i filtri di ricerca.
	
	\paragraph{Design pattern:} Facade, Singleton.

	\subsection[FilterView]{FilterView | Filtro di ricerca}
	\label{sec:stage:design:sistema:view.filter:filter}
	Tale componente rappresenta l'interfaccia delle componenti grafiche che rappresentano le tipologie standard dei filtri per il raffinamento dei risultati di una ricerca, ossia (\textit{\nameref{sec:stage:design:sistema:view.filter:list-filter}}, \textit{\nameref{sec:stage:design:sistema:view.filter:range-filter}}, \textit{\nameref{sec:stage:design:sistema:view.filter:switch-filter}} e \textit{\nameref{sec:stage:design:sistema:view.filter:value-filter}}).

	\subsection[ListFilterView]{ListFilterView | Filtro con lista di valori}
	\label{sec:stage:design:sistema:view.filter:list-filter}
	Il componente grafico - associato alla classe \textit{\nameref{sec:stage:design:sistema:model.filter:list-filter}} - visualizza le liste dei valori ammissibili e bloccati e consente all'utente di:
	\begin{itemize}
	\item spostare un valore da una lista all'altra;
	\item azzerare il filtro, spostando automaticamente tutti i valori nella lista degli ammissibili.
	\end{itemize}

	\subsection[RangeFilterView]{RangeFilterView | Filtro con intervallo di valori}
	\label{sec:stage:design:sistema:view.filter:range-filter}
	Il componente grafico - associato alla classe \textit{\nameref{sec:stage:design:sistema:model.filter:range-filter}} - permette all'utente di specificare l'estremo inferiore e superiore dell'intervallo dei valori ammissibili secondo le regole definite nella suddetta classe.

	\subsection[SwitchFilterView]{SwitchFilterView | Filtro a doppio stato}
	\label{sec:stage:design:sistema:view.filter:switch-filter}
	Il componente grafico consente di abilitare o disattivare un filtro di tipo \textit{\nameref{sec:stage:design:sistema:model.filter:switch-filter}} e ne visualizza lo stato corrente.

	\subsection[ValueFilterView]{ValueFilterView | Filtro con soglia di valore}
	\label{sec:stage:design:sistema:view.filter:value-filter}
	Il componente grafico - associato alla classe \textit{\nameref{sec:stage:design:sistema:model.filter:value-filter}} - consente all'utente di modificare il valore della soglia secondo le condizioni previste dalla suddetta classe.

	% SECTION
	\section{Package view.search}
	\label{sec:stage:design:sistema:view.search}

	\subsection[EntityList]{EntityList | Elenco delle entità cercate}
	\label{sec:stage:design:sistema:view.search:search-entity-list}
	Al termine della ricerca, tale componente visualizza le entità (\textit{\nameref{sec:stage:design:sistema:view.search:entity}}) riferite dalle etichette e cercate tra i contenuti.

	\subsection[EntityView]{EntityView | Entità}
	\label{sec:stage:design:sistema:view.search:entity}
	Tale componente grafico visualizza le informazioni essenziali associate ad un'entità e consente all'utente di:
	\begin{itemize}
	\item visualizzare la lista delle entità padre;
	\item visualizzare la lista delle entità figlie;
	\item sostituire l'entità corrente con un padre o un figlio.
	\item eliminare l'entità corrente.
	\end{itemize}

	\subsection[SearchBar]{SearchBar | Barra di ricerca}
	\label{sec:stage:design:sistema:view.search:search-bar}
	La barra di ricerca rappresenta il componente grafico (campo di testo) mediante il quale l'utente può inserire i termini di ricerca.

	\subsection[SearchScopeSelector]{SearchScopeSelector | Selettore dell'ambito di ricerca}
	\label{sec:stage:design:sistema:view.search:search-scope-selector}
	Il selettore dell'ambito di ricerca è un componente grafico (menu a tendina) che permette all'utente di circoscrivere la ricerca ad informazioni specifiche:
	\begin{description}
	\item[Tutto] \hfill \\
	Estende la ricerca a tutti i tipi di informazioni indicizzate o ricercabili all'interno della piattaforma.
 	\item[Etichette] \hfill \\
	Limita la ricerca alle sole etichette assegnate ai contenuti.
	\item[Frasi] \hfill \\
	Limita la ricerca alle informazioni presenti in un contenuto (titolo, corpo, \ldots).
	\end{description}

	\subsection[LabelEntityList]{LabelEntityList | Accezioni di un'etichetta}
	\label{sec:stage:design:sistema:view.search:label-entity-list}
	Ove il sistema individui tra i termini cercati delle etichette aventi accezioni multiple. tale componente grafico provvedere a mostrare - per ciascuna etichetta - la lista delle entità (\textit{\nameref{sec:stage:design:sistema:view.search:entity}}) riferite e consente all'utente di selezionare quella rispetto cui intenda procedere con la ricerca.

	Al termine della ricerca, l'entità selezionata viene mostrata nel componente \textit{\nameref{sec:stage:design:sistema:view.search:search-entity-list}}.

	% SECTION
	\section{Package view.timeline}
	\label{sec:stage:design:sistema:view.timeline}

	%----------
	% CAPITOLO
	%----------
	\chapter{Componente Controller}
	\label{ch:stage:design:controller}
	Questo capitolo illustra i componenti del \textit{Controller}, per ciascuno dei quali è indicato il nome della classe accompagnata da un'identificazione sintetica, separati dal carattere '|' (separatore verticale).

	% SECTION
	\section{Package controller}
	\label{ch:stage:design:sistema:controller}

	% SECTION
	\section{Package controller.filter}
	\label{ch:stage:design:sistema:controller.filter}

	% SECTION
	\section{Package controller.search}
	\label{ch:stage:design:sistema:controller.search}

\end{document}
