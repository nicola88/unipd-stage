\documentclass[10pt,a4paper,headinclude,footinclude,hidelinks]{scrreprt} % KOMA-Script
\usepackage[italian]{babel}
\usepackage[utf8x]{inputenc}
\usepackage[]{../../classicthesis} % nochapters
\pagestyle{scrheadings}
\setcounter{tocdepth}{1}

\begin{document}
    \title{\rmfamily\normalfont\spacedallcaps{Sistema di classificazione}}
    \author{\spacedlowsmallcaps{Nicola Moretto (matr. 578258)}}
    \date{\today}
    
    \maketitle
    
    \begin{abstract}
        \noindent Il documento presenta i risultati delle fasi di analisi e di progettazione dei nuovi criteri di classificazione.
    \end{abstract}
    
	\begin{table}[ht]
	\centering
	\begin{tabular}{|c|c|l|}
	\hline
	\textsc{Versione} & \textsc{Data} & \textsc{Modifiche} \\ \hline
	0.1 & 10-09-2012 & Prima stesura del documento. \\ \hline
	0.2 & 11-09-2012 & Aggiunto il capitolo \nameref{ch:stage:cls:contenuti}. \\ \hline
	0.3 & 12-09-2012 & Aggiunto il capitolo \nameref{ch:stage:cls:scenari}. \\ \hline
	0.4 & 13-09-2012 & Ampliato il capitolo \nameref{ch:stage:cls:scenari}. \\ \hline
	0.5 & 14-09-2012 & Aggiunto il capitolo \nameref{ch:stage:cls:requisiti}. \\ \hline
	\end{tabular}
	\caption{Registro delle modifiche}
	\label{tab:stage:wp:workload}
	\end{table}

	\tableofcontents
	\listoffigures
	\begingroup
	\let\clearpage\relax
	\listoftables
	\endgroup

	\chapter{Contenuti informativi}
	\label{ch:stage:cls:contenuti}
	\section{Classi di contenuti}
	\subsection{Documento}
	\subsection{Domanda}
	\subsection{Evento}
	\subsection{Multimedia}
	\subsection{Pensiero}
	\subsection{Risposta}
	\label{sec:stage:cls:contenuti:classi}
	\section{Relazioni tra contenuti}
	\label{sec:stage:cls:contenuti:relazioni}
	
	
	\chapter{Scenari d'uso}
	\label{ch:stage:cls:scenari}
	\section{Gestione delle etichette}
	\label{sec:stage:cls:scenari:etichette}
	\subsection{Aggiunta di un sinonimo ad un'etichetta primaria}
	Gli utenti possono scegliere etichette\marginpar{Duplicità sintattica} differenti per riferire la stessa entità (concreta o astratta): ove si trascurino i legami sinonimici tra le etichette, tale duplicità determina la parzialità dei risultati di una ricerca sui contenuti a seconda dell'etichetta scelta, essendo restituito il sottoinsieme di contenuti nei quali l'entità sia identificata precisamente da tale etichetta.

	L'esito desiderato della ricerca\marginpar{Sintassi e semantica} consiste invece nell'insieme di contenuti in cui l'entità sia riferita, a prescindere dalla specifica etichetta utilizzata: in altre parole, si desidera che la ricerca venga trasferita dal piano puramente sintattico (l'etichetta specifica) a quello semantico (l'entità indicata dall'etichetta).

	Per evitare la proliferazione di etichette duplicate\marginpar{Entità e etichette}, ossia sintatticamente differenti ma riferenti la medesima entità, è utile, per ogni entità, 1) definire un'etichetta che la identifichi chiaramente all'interno della piattaforma e 2) tenere traccia dei sinonimi utilizzati dagli utenti per riferire tale entità.

	I sinonimi vengono quindi mantenuti per garantire una maggior \marginpar{Copertura sintattica}copertura sintattica, aiutando a stabilire se un'etichetta cercata o scelta dall'utente sia già presente nel dizionario, ma il loro impiego è scoraggiato a favore delle corrispettive chiavi primarie, al fine di rendere più efficiente la catalogazione, la ricerca, la navigazione e la consultazione dei contenuti stessi.

	Tali esigenze ci portano a distinguere due classi di etichette\marginpar{Classi di etichette}:
	\begin{description}
	\item[Etichetta primaria] \hfill \\
	Ciascuna etichetta primaria identifica univocamente un'entità specifica del dominio.
	\item[Etichetta sinonimica] \hfill \\
	Ciascuna etichetta sinonimica rappresenta un sinonimo per una (e una sola) etichetta primaria. 
	\end{description}

	\subsection{Aggiunta di un'accezione ad un'etichetta}
	Ciascuna etichetta\marginpar{Ambiguità semantica} può riferirsi a entità differenti a seconda del contesto in cui viene utilizzata: nell'ambito della piattaforma, diventa cruciale poter precisare facoltativamente l'accezione dell'etichetta (e quindi l'entità specifica) rispetto alla quale si vuole effettuare una ricerca sui contenuti informativi.

	Ad ogni etichetta è possibile associare\marginpar{Accezioni}, sotto forma di metadati, diverse accezioni, ciascuna delle quali identifica univocamente un possibile significato o entità.
	\subsection{Aggiunta di un'etichetta}
	Il dizionario\marginpar{Partizione} rappresenta - nella visione più elementare - un insieme di etichette primarie e sinonimi: più in dettaglio, è possibile immaginarlo come una partizione, ciascuno dei cui sottoinsiemi corrisponde ad un'entità del dominio e contiene esattamente un'etichetta primaria, che identifica univocamente il sottoinsieme/entità, e gli eventuali sinonimi (in numero arbitrario, anche nullo).
	
	L'aggiunta di una nuova etichetta\marginpar{Inserimento} consiste innanzi tutto nello stabilire se si tratti di:
	\begin{description}
	\item[Etichetta primaria] \hfill \\
	La nuova etichetta contribuisce a formare un sottoinsieme logico distinto, contenete un unico elemento (la chiave stessa).
 	\item[Etichetta sinonimica] \hfill \\
	La nuova etichetta viene inserita nel sottoinsieme logico identificato dall'unica etichetta primaria di cui è sinonimo.
	\end{description}
	
	Non si da il caso che\marginpar{Uno-a-molti} una nuova etichetta $S_1$ possa essere sinonimo di due (o più) etichette primarie. Si considerino ad esempio due etichette primarie, $E_1$ e $E_2$: per la proprietà transitiva, se $E_1$ è sinonimo di $S_1$ e $E_2$ è sinonimo di $S_1$, allora $E_1$ e $E_2$ sono a loro volta sinonimi; ma allora, in accordo ai principi sopra illustrati, l'ultima tra $E_1$ e $E_2$ ad essere stata aggiunta doveva essere inserita nel sottoinsieme dell'altra, contraddicendo le ipotesi iniziali.

	\subsection{Eliminazione di un'etichetta}
	In maniera simile a quanto visto per l'aggiunta di un'etichetta, occorre distinguere due casi, a seconda che si tratti di:
	\begin{description}
	\item[Etichetta primaria] \hfill \\
	L'etichetta viene eliminata e un sinonimo dell'etichetta viene promosso a etichetta primaria.
 	\item[Etichetta sinonimica] \hfill \\
	Il sinonimo viene semplicemente eliminato.
	\end{description}
	
	\section{Inserimento di un contenuto}
	\subsection{Assegnazione di un'etichetta}
	L'assegnazione di un'etichetta ad un contenuto consiste nell'individuazione di parole o brevi espressioni chiave, che identifichino un'entità concreta (luogo, persona, oggetto, \ldots) o astratta (concetto, argomento, \ldots) riferita o citata all'interno del contenuto stesso.
	
	Una volta individuata la parola o espressione\marginpar{Etichetta esistente}, il sistema deve verificare se essa sia già stata utilizzata in precedenza e quindi presente nel dizionario interno: in caso affermativo, possono verificarsi due casi:
	\begin{description}
	\item[Etichetta primaria] \hfill \\
	L'etichetta viene associata al contenuto.
	\item[Etichetta sinonimica] \hfill \\
	L'etichetta viene automaticamente rimpiazzata con la corrispondente etichetta primaria.
	\end{description}

	In caso contrario\marginpar{Nuova etichetta}, viene indagata la presenza nel dizionario interno di etichette aventi significato analogo a quella prescelta dall'utente: a seconda dell'esito della ricerca possono verificarsi due casi:
	\begin{description}
	\item[Nessun risultato] \hfill \\
	La parola o espressione viene memorizzata nel dizionario come etichetta primaria.
	\item[Etichetta primaria] \hfill \\
	La parola o espressione viene memorizzata nel dizionario come sinonimo dell'etichetta primaria.
	\end{description}

	In ogni caso, al termine della procedura al contenuto viene assegnata un'etichetta primaria e l'utente ha facoltà di specificare - ove disponibile - un'accezione.
	\subsection{Rimozione di un'etichetta}
	La rimozione di un'etichetta assegnata in precedenza ad un contenuto non modifica in alcun modo il dizionario interno, anche qualora non risultasse assegnata ad altri contenuti.

	\section{Consultazione dei contenuti}
	\label{sec:stage:cls:scenari:read}
	La ricerca e la consultazione dei contenuti rappresentano attività cruciali per gli utenti della piattaforma e ci si affida al criterio di classificazione delle etichette per reperire in maniera efficiente le informazioni cercate; l'approccio e lo scopo con cui gli utenti navigano l'insieme di contenuti disponibili all'interno della piattaforma può tuttavia differire sensibilmente.
	\subsection{Esplorazione delle etichette}
	L'utente alle prime armi\marginpar{Gerarchia} o semplicemente interessato a conoscere gli argomenti discussi nella piattaforma esplora i contenuti informativi a partire dalle etichette: per facilitare la navigazione di queste ultime dovrebbe essere disponibile una struttura gerarchica, che le raccolga e le cataloghi in maniera ordinata.

	Tale soluzione\marginpar{Dicotomia} permette all'utente di individuare più rapidamente gli argomenti di interesse mediante un \textsc{processo dicotomico}, che partendo dagli argomenti generali proceda per ulteriori raffinamenti ad individuare con crescente precisione e accuratezza i temi di interesse, senza dover consultare l'insieme delle etichette nella sua completezza.

	Gli elementi della gerarchia\marginpar{Elementi} sono etichette primarie, i cui sinonimi e accezioni vengono mostrati come metadati associati. 
	\subsection{Ricerca di un'etichetta}
	L'utente alla ricerca di informazioni su un tema particolare cerca di individuare le etichette esistenti aventi maggiore attinenza e rilevanza con il tema in questione: la ricerca di corrispondenze nel dizionario prevede che:
	\begin{enumerate}
	\item vengano prese in esame tutte le etichette, poiché contemplando non solo le chiavi primarie ma anche i relativi sinonimi aumenta la probabilità di ottenere riscontri positivi (maggiore copertura sintattica);
	\item vengano restituite le chiavi primarie corrispondenti alla ricerca: qualora si abbia riscontro su un sinonimo, viene restituita la corrispondente chiave primaria.
	\end{enumerate}

	%\subsection{Ricerca di etichette correlate}
	\subsection{Ricerca contenuti per etichetta}
	La ricerca di informazioni\marginpar{Etichette e accezioni} riguardanti un tema specifico viene effettuata specificando una o più etichette, eventualmente declinate nelle specifiche accezioni, che presentino agli occhi dell'utente particolare attinenza con il tema in questione e siano dunque con maggior probabilità associate ai contenuti di interesse.

	Siano $U_e$ l'insieme delle etichette\marginpar{Insiemi di etichette} inserite e $C_e$ l'insieme delle etichette assegnate ad un generico contenuto: il primo passo consiste nel sostituire le etichette sinonimiche con le equivalenti primarie ed estendere l'insieme $U_e$ alle etichette figlie di ogni $e \in U_e$.

	A questo punto\marginpar{Corrispondenza} si possono distinguere tre casi principali, a seconda del grado di corrispondenza/attinenza dei contenuti rispetto alle etichette cercate:
	\begin{description}
	\item[Corrispondenza completa:] $U_e \subseteq C_e$ \hfill \\
	Al contenuto sono state assegnate tutte le etichette richieste dall'utente e verrà quindi visualizzato in cima ai risultati di ricerca (massima attinenza).
 	\item[Corrispondenza parziale:] $U_e \cap C_e \neq \emptyset$ \hfill \\
	Al contenuto sono state assegnate parte delle etichette richieste dall'utente (media attinenza).
	\item[Nessuna corrispondenza:] $U_e \cap C_e = \emptyset$\hfill \\
	Al contenuto non risulta assegnata alcuna etichetta richiesta dall'utente (attinenza nulla).
	\end{description}

	I contenuti attinenti\marginpar{Attinenza} vengono visualizzati in ordine decrescente rispetto al numero di etichette assegnate corrispondenti a quelle richieste dall'utente:
	$$\left|{U_e \cap C_e}\right|$$
	\subsection{Ricerca di contenuti affini}
	La ricerca di contenuti affini consiste nell'identificare, a partire da un contenuto dato, altri la cui pertinenza rispetto al tema trattato sia massima: in questo scenario valgono le medesime considerazioni emerse nella sezione precedente, previa sostituzione di $U_e$ con l'insieme delle etichette assegnate al contenuto corrente.
	
	\chapter{Requisiti}
	\label{ch:stage:cls:requisiti}
	\section{Gerarchia delle etichette}
	

	%\chapter{Progettazione}

\end{document}
