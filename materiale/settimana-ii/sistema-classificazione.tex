\documentclass[10pt,a4paper,headinclude,footinclude,hidelinks]{scrreprt} % KOMA-Script
\usepackage[italian]{babel}
\usepackage[utf8]{inputenc}
\usepackage[T1]{fontenc}
\usepackage{graphicx}
\usepackage{amsfonts}
\usepackage[]{../../classicthesis} % nochapters
\pagestyle{scrheadings}
\setcounter{tocdepth}{2}

\begin{document}
    \title{\rmfamily\normalfont\spacedallcaps{Sistema di classificazione}}
    \author{\spacedlowsmallcaps{Nicola Moretto (matr. 578258)}}
    \date{\today}
    
    \maketitle
    
    \begin{abstract}
        \noindent Il documento presenta i risultati delle fasi di analisi e di progettazione dei nuovi criteri di classificazione.
    \end{abstract}
    
	\begin{table}[ht]
	\centering
	\begin{tabular}{|c|c|l|}
	\hline
	\textsc{Versione} & \textsc{Data} & \textsc{Modifiche} \\ \hline
	0.1 & 10-09-2012 & Prima stesura del documento. \\ \hline
	0.2 & 11-09-2012 & Aggiunto il capitolo \nameref{ch:stage:contenuti}. \\ \hline
	0.3 & 12-09-2012 & Aggiunto il capitolo \nameref{ch:stage:req}. \\ \hline
	0.4 & 13-09-2012 & Ampliato il capitolo \nameref{ch:stage:req}. \\ \hline
	0.5 & 14-09-2012 & Rivisto il capitolo \nameref{ch:stage:req}. \\ \hline
	1.0 & 15-09-2012 & Pubblicazione della prima versione ufficiale. \\ \hline
	1.1 & 18-09-2012 & Revisione e ampliamento della sezione \nameref{ch:stage:req}. \\ \hline
	\end{tabular}
	\caption{Registro delle modifiche}
	\label{tab:stage:wp:workload}
	\end{table}

	\tableofcontents

	\chapter{Contenuti informativi}
	\label{ch:stage:contenuti}
	\section{Introduzione}
	Il patrimonio di conoscenza della piattaforma è garantito essenzialmente e primariamente dai contenuti pubblicati dagli utenti, che condividono alcune proprietà essenziali (autore, data di pubblicazione, visibilità, \ldots) e un contenuto informativo vero e proprio, di lunghezza (massima) variabile.

	Le classi di contenuti\marginpar{Classi} rispecchiano altrettante forme di espressione quotidiana (la domanda, il pensiero elementare, un discorso articolato, \ldots), facilmente riconoscibili da qualsiasi utente, e di contenuto (audio, video, evento, \ldots).

	\section{Criteri di classificazione}
	Per facilitare la catalogazione e il reperimento dei contenuti, essi condividono, a prescindere dalla rispettiva classe, i medesimi criteri di classificazione, ciascuno dei quali ne valuta e pesa un aspetto differente:
	\subsection{Argomento}
	Branca del sapere - agnostica rispetto al tema specifico della piattaforma - entro la quale ciascun contenuto della piattaforma si colloca univocamente.
	\subsection{Emozione}
	Emozioni personali che l'autore associa al contenuto.
	\subsection{Intenzioni}
	Intenzioni con cui l'autore scrive il contenuto (opinione, critica, \ldots) e utili a chiarire lo spirito con cui debba essere interpretato.
	\subsection{Giudizi}
	Giudizi qualitativi espressi dagli altri utenti su un contenuto. I criteri e i parametri con cui tali valutazioni verranno espresse sono attualmente in fase di indagine da parte di altri membri del team di progetto.

	\section{Classi}

	\subsection{Documento}
	La classe \textsc{documento} è concepita per esprimere un contenuto prevalentemente testuale, di lunghezza rilevante e articolato nella struttura; al suo interno l'utente può esporre delle tesi o opinioni, supportandole con opportune argomentazioni, notizie dettagliate, \ldots.
	\subsection{Domanda}
	La classe \textsc{domanda} offre la possibilità di sottoporre agli utenti della piattaforma una domanda relativa ad un certo tema o ad un contenuto specifico.
	\subsection{Evento}
	La classe \textsc{evento} permette di pubblicizzare un evento o manifestazione, indicandone luogo e data di svolgimento, se sia pubblico o privato, \ldots.
	\subsection{Multimedia}
	La classe \textsc{multimedia} consente di pubblicare contenuti audio e video, sia in risposta sia in forma completamente autonoma rispetto ad altri contenuti informativi.
	\subsection{Pensiero}
	La classe \textsc{pensiero} è concepita per esprimere idee, concetti o pensieri semplici ed essenziali, la cui lunghezza risulta dunque limitata.
	\subsection{Risposta}
	La classe \textsc{risposta} offre la possibilità di inserire una risposta ad una domanda precedente o un commento ad un generico contenuto.
	
	\section{Relazioni}
	\label{sec:stage:cls:contenuti:relazioni}
	All'interno della piattaforma il generico contenuto\marginpar{Contenuto generico} riveste un ruolo essenziale rappresentando l'astrazione fondamentale su cui poggiano tutti i tipi di contenuti e sulla quale è definita la maggior parte delle relazioni, sia interne (tra i contenuti stessi) sia esterne (criteri di classificazione, \ldots).

%	\begin{figure}[ht]
%		\begin{center}
%	    	\includegraphics{../img/placeholder.png}
%			\caption{Flusso di contenuti}
%		\end{center}
%	\end{figure}

	A ciascun contenuto\marginpar{Discussione} pubblicato nella piattaforma è possibile rispondere con altri del medesimo tipo o differente: ciò implica che, a partire da un contenuto qualsiasi, può nascere una discussione in grado di svilupparsi e ramificarsi con il massimo grado di libertà, non essendovi limiti sui tipi di contenuti o vincoli sul tema. Ad esempio, una risposta ad un contenuto può - in virtù di una particolare associazione di idee - riguardare un tema non strettamente correlato al contenuto di partenza.
	

	\chapter{Requisiti}
	\label{ch:stage:req}

	Introduzione sui contenuti della piattaforma e le entità (multiple) da riferire.

	\section{Entit\`a}
	\label{sec:stage:req:entità}
	\subsection{Identificazione univoca di un'entit\`a}
	Entità $\to$ identificatore
	\subsection{Identificazione non ambigua di un'entit\`a}
	Identificatore $\to$ entità
	\subsection{Gestione delle relazioni tra entit\`a}

	Riprendendo la similitudine tra il dizionario della etichette e uno linguistico, si giunge facilmente a notare come non si tratti, in entrambi i casi, di semplici insieme di entit $E_i$ slegate, ma si possano intuitivamente individuare tra di esse delle relazioni gerarchiche (dal generale al particolare), capaci di trascendere la banale enumerazione alfabetica.

	Nel dizionario della piattaforma\marginpar{Gerarchia} tali relazioni vengono rese attraverso legami tra le etichette primarie, ciascuna delle quali identifica univocamente un'entità del dominio ($e_0 \in E_i$), e si traducono nella possibilità di associare a ciascuna etichetta primaria un numero arbitrario di padri (etichette generiche) e figli (etichette specialistiche).

	A differenze delle strutture gerarchiche classiche\marginpar{Estensioni}, ove ciascun elemento può avere molti figli ma un solo padre, il dizionario delle etichette estende la relazione \textit{uno-a-molti} anche agli elementi padre: la possibilità di assegnare differenti padri ad una medesima etichetta primaria si spiega con il desiderio di eliminare eventuali e possibili ambiguità, consentendo ad un utente di individuare un'etichetta seguendo differenti cammini nella gerarchia.

	\subsubsection{Ciascuna entit\`a ha $0\ldots n$ figli}
	\subsubsection{Ciascuna entit\`a ha $0\ldots n$ padri}
	\subsubsection{Principio di sostituzione}
	\subsection{Ricerca di un'entit\`a}

	\paragraph{Soluzione}
	\begin{enumerate}
	\item Distinzione tra etichette primarie e sinonimiche (rispetto ad accezioni di un'etichetta).
	\item Relazione 1:1 tra etichette primarie ed entità.
	\item Grafo aciclico orientato delle entità.
	\end{enumerate}

	% SEZIONE 
	\section{Etichette}
	\label{sec:stage:req:tag}
	
	Il \textsc{dizionario} della piattaforma\marginpar{Modello concettuale} rappresenta - nella visione più elementare - un insieme di \textsc{etichette}: più in dettaglio, è possibile paragonarlo ad un dizionario linguistico, costituito da una insieme di lemmi, ciascuno dei quali possiede svariati significati (\textsc{accezioni}), che - a seconda del contesto o ambito di riferimento - identificano differenti entità del dominio.

Riprendendone la connotazione insiemistica\marginpar{Notazione}, può risultare conveniente immaginare e trattare il dizionario $D$ come l'unione di $n$ sottoinsiemi $E_i$, ciascuno dei quali corrisponde ad un'entità del dominio e contiene esattamente un'etichetta \textsc{primaria} $e_0$, che identifica univocamente il sottoinsieme o entità in questione, e gli eventuali $m$ \textsc{sinonimi} $e_j$ (in numero arbitrario, anche nullo).\footnote{$i \in \mathbb{N}, i \leq n=\left|D\right|$}\footnote{$j \in \mathbb{N}, j \leq m=\left|E_i\right|$}

%	\begin{figure}[ht]
%		\begin{center}
%	    	\includegraphics{../img/placeholder.png}
%			\caption{Dizionario delle etichette}
%		\end{center}
%	\end{figure}

	\subsection{Gestione dei sinonimi}
	\subsubsection{Ciascuna etichetta può avere $0\ldots n$ sinonimi}
	Gli utenti possono scegliere etichette\marginpar{Ambiguità sintattica} differenti per riferire la stessa entità (concreta o astratta): trascurando i legami sinonimici tra le etichette, tale ambiguità determina la parzialità dei risultati di una ricerca a seconda dell'etichetta scelta, poiché verrebbe restituito il sottoinsieme dei contenuti all'interno dei quali l'entità sia identificata precisamente da tale etichetta.
	
%	\begin{figure}[ht]
%		\begin{center}
%	    	\includegraphics{../img/placeholder.png}
%			\caption[Ambiguità sintattica]{Ambiguità sintattica: un significante, molti significati}
%		\end{center}
%	\end{figure}

	L'esito desiderato della ricerca\marginpar{Sintassi e semantica} consiste invece nell'insieme di contenuti in cui l'entità sia riferita, a prescindere dalla specifica etichetta utilizzata: in altre parole, si desidera che la ricerca venga trasferita dal piano puramente sintattico (l'etichetta specifica) a quello semantico (l'entità indicata dall'etichetta).

	Per evitare la proliferazione di etichette duplicate\marginpar{Etichette primarie e sinonimiche} (sintatticamente differenti ma riferenti la medesima entità), che contribuirebbe a indebolire l'efficacia (qualità dei risultati di ricerca, navigabilità dei contenuti, \ldots) e l'efficienza (dimensione del dizionario, \ldots) del sistema di classificazione, risulta utile, per ogni entità $E_i$:
	\begin{enumerate}
	\item definire un'etichetta che la identifichi chiaramente all'interno della piattaforma (\textsc{etichetta primaria} $e_0$);
	\item tenere traccia dei sinonimi utilizzati dagli utenti per riferire tale entità (\textsc{etichette sinonimiche} $e_j$).
	\end{enumerate}

%	\begin{figure}[ht]
%		\begin{center}
%	    	\includegraphics{../img/placeholder.png}
%			\caption{Etichette primarie e sinonimiche}
%		\end{center}
%	\end{figure}

	Le etichette sinonimiche\marginpar{Copertura sintattica} vengono conservate nel dizionario per garantire una maggior copertura sintattica, aiutando a stabilire se un'etichetta cercata o scelta dall'utente sia già stata utilizzata in precedenza (con quale significato, in quale ambito, \ldots) e dunque risulti già presente nel dizionario.

	Ai contenuti tuttavia vengono assegnate le corrispettive chiavi primarie, al fine di rendere più efficiente la catalogazione, la ricerca, la navigazione e la consultazione dei contenuti stessi.

	\subsubsection{Aggiunta di un sinonimo ad un'etichetta}
	Ogni qualvolta un utente sceglie una nuova etichetta $e$, che risulti sinonimo di un'altra esistente $e_j \in E_i$, essa viene aggiunta al dizionario interno della piattaforma come $e_{m+1} \in E_i$ sinonimo di $e_0 \in E_i$.

	Da quel momento, qualora un utente provi ad assegnare quella medesima parola o espressione ad un contenuto della piattaforma, il sistema vi 
assegnerà automaticamente l'etichetta primaria $e_0$.

%	L'aggiunta di un'etichetta sinonimica consiste nell'assegnare ad un'entità esistente un nuovo possibile identificatore, associando tale etichetta $e_{m+1} \in E_i$ alla corrispondente primaria $e_0 \in E_i$. Da un punto di vista insiemistico si assiste all'inserimento dell'etichetta medesima nel sottoinsieme corrispondente alla suddetta entità e identificato dall'etichetta primaria $e_0$.

%	Non si da il caso che la nuova etichetta\marginpar{Accezioni e sinonimi} $e_{m+1}$ possa essere sinonimo - rispetto ad una specifica accezione - di due (o più) etichette primarie, ma può essere sinonimo di etichette primarie in numero al più pari alle relative accezioni.

%	Si considerino ad esempio\marginpar{Uno-a-molti} due etichette primarie, $e_1 \in E_i$ e $e_2 \in E_i$: per la proprietà transitiva, se $e_1$ è sinonimo di $e_{m+1}$ e $e_2$ è sinonimo di $e_{m+1}$, allora $e_1$ e $e_2$ sono a loro volta sinonimi; ma allora, in accordo ai principi sopra illustrati, l'ultima tra $e_1$ e $e_2$ ad essere stata aggiunta doveva essere inserita nel sottoinsieme dell'altra, contraddicendo così le ipotesi iniziali.

	\subsubsection{Eliminazione di un sinonimo associato ad un'etichetta}

	\subsection{Gestione delle accezioni}
	\subsubsection{Ciascuna etichetta può avere $1\ldots n$ accezioni}
	Ciascuna etichetta\marginpar{Ambiguità semantica} può riferirsi a entità differenti a seconda del contesto, perciò diventa cruciale poterne precisare le possibili accezioni $a_k \in A$, ossia le entità cui si possa riferire.\footnote{$k \in \mathbb{N}, k \leq t=\left|A\right|$}
%	\begin{figure}[ht]
%		\begin{center}
%	    	\includegraphics{../img/placeholder.png}
%			\caption[Ambiguità semantica]{Ambiguità semantica: un significato, molti significanti}
%		\end{center}
%	\end{figure}
	
	Con l'introduzione delle accezioni\marginpar{Accezioni, entità e sottoinsiemi}, il dizionario della piattaforma acquisisce una nuova dimensione poiché ciascuna etichetta può appartenere contemporaneamente a diversi sottoinsiemi: dal momento che ciascuna entità corrisponde ad un sottoinsieme distinto e ogni accezione di un'etichetta riferisce un'entità diversa, la medesima etichetta si colloca in $\left|A\right|$ sottoinsiemi differenti.
%	\begin{figure}[ht]
%		\begin{center}
%	    	\includegraphics{../img/placeholder.png}
%			\caption[Accezioni e sottoinsiemi]{Entità distinte: accezioni e sottoinsiemi di un'etichetta}
%		\end{center}
%	\end{figure}

	\subsubsection{Aggiunta di un'accezione ad un'etichetta}
	L'aggiunta di un'accezione ad un'etichetta consiste nel definire il contesto o ambito in cui essa assuma un significato univoco e non equivocabile.

	\subsubsection{Eliminazione di un'accezione associata ad un'etichetta}
	L'eliminazione di un'accezione $a_k \in A_j$ associata ad un'etichetta $e_j \in E_i$ prevede due possibili casi:
	\begin{description}
	\item[Etichetta primaria:] \hfill \\
	Se l'etichetta è primaria, l'accezione viene eliminata e un sinonimo viene promosso in sua vece ad etichetta primaria del sottoinsieme.
 	\item[Etichetta sinonimica] \hfill \\
	Se l'etichetta è sinonimica, si procede direttamente alla cancellazione dell'accezione	.
	\end{description}

	\subsection{Gestione del dizionario}
	Il dizionario contiene in ogni istante $$\sum_{i \in \mathbb{N}, i \leq n}\left|E_i\right|$$ etichette, a ciascuna delle quali sono associate $\left|A_{i,j}\right|$ accezioni.

	\subsubsection{Il dizionario contiene $0\ldots n$ etichette}

	\subsubsection{Inserimento di una nuova etichetta}
	L'aggiunta di un'etichetta primaria consiste nell'individuare una nuova entità ancora ignota al dizionario e nell'assegnarle come identificatore tale etichetta. Da un punto di vista insiemistico, si assiste alla formazione di un nuovo sottoinsieme $E_i$ contenente un unico elemento $e_0$, ossia l'etichetta stessa.

	\subsubsection{Eliminazione di un'etichetta esistente}
	L'eliminazione di un'etichetta $e_j \in E_i$ richiede di considerare separatamente ogni possibile accezione $a_k \in A_j$, valutando caso per caso:
	\begin{description}
	\item[Etichetta primaria]\hfill \\
	Se l'etichetta è primaria viene eliminata e un sinonimo viene promosso in sua vece ad etichetta primaria del sottoinsieme.
 	\item[Etichetta sinonimica] \hfill \\
	Se l'etichetta è sinonimica si procede semplicemente alla sua cancellazione.
	\end{description}

	\section{Contenuti} % Inserimento vs consultazione
	
	\subsection{Gestione delle etichette}
	\subsubsection{A ciascun contenuto possono essere assegnate $0\ldots n$ etichette}
	\subsubsection{Assegnazione di un'etichetta ad un contenuto}
	L'assegnazione di un'etichetta ad un contenuto consiste nell'individuazione di parole o brevi espressioni chiave, che identifichino un'entità concreta (luogo, persona, oggetto, \ldots) o astratta (concetto, argomento, \ldots) riferita o citata all'interno del contenuto stesso.
	
	Una volta individuata la parola o espressione\marginpar{Etichetta esistente}, il sistema deve verificare se essa sia già stata utilizzata in precedenza e quindi presente nel dizionario interno: in caso affermativo, possono verificarsi due casi:
	\begin{description}
	\item[Etichetta primaria] \hfill \\
	L'etichetta viene associata al contenuto.
	\item[Etichetta sinonimica] \hfill \\
	L'etichetta viene automaticamente rimpiazzata con la corrispondente etichetta primaria.
	\end{description}

	In caso contrario\marginpar{Nuova etichetta}, viene indagata la presenza nel dizionario interno di etichette aventi significato analogo a quella prescelta dall'utente: a seconda dell'esito della ricerca possono verificarsi due casi:
	\begin{description}
	\item[Nessun risultato] \hfill \\
	La parola o espressione viene memorizzata nel dizionario come etichetta primaria.
	\item[Etichetta primaria] \hfill \\
	La parola o espressione viene memorizzata nel dizionario come sinonimo dell'etichetta primaria.
	\end{description}

	In ogni caso, al termine della procedura al contenuto viene assegnata un'etichetta primaria e l'utente ha facoltà di specificare - ove disponibile - un'accezione.
	\subsubsection{Eliminazione di un'etichetta associata ad un contenuto}
	La rimozione di un'etichetta assegnata in precedenza ad un contenuto non modifica in alcun modo il dizionario interno, anche qualora tale etichetta non risultasse assegnata ad altri contenuti.

	\subsection{Ricerca e navigazione}
	La ricerca e la consultazione dei contenuti rappresentano attività cruciali per gli utenti della piattaforma e ci si affida al criteri di classificazione delle etichette per reperire in maniera efficiente le informazioni cercate; l'approccio e lo scopo con cui gli utenti navigano l'insieme di contenuti disponibili all'interno della piattaforma può tuttavia differire sensibilmente.

	\subsubsection{Ricerca di contenuti generici}
	L'utente alle prime armi\marginpar{Gerarchia} o semplicemente interessato a conoscere gli argomenti discussi nella piattaforma esplora i contenuti informativi a partire dalle etichette: per facilitarne la navigazione si ricorre ad una struttura gerarchica, che le raccolga e le cataloghi in maniera ordinata.

	Tale soluzione\marginpar{Dicotomia} permette all'utente di individuare più rapidamente gli argomenti di interesse mediante un \textsc{processo dicotomico}, che partendo dagli argomenti generali proceda per ulteriori raffinamenti sino ad individuare con crescente precisione e accuratezza i temi di interesse, senza dover consultare esaustivamente l'insieme delle etichette.

	Gli elementi della gerarchia\marginpar{Elementi} sono etichette primarie, a ciascuna delle quali sono associate delle accezioni e - per ciascuna di esse - i relativi sinonimi. 
	
	\paragraph{Ricerca di un'etichetta}
	L'utente alla ricerca di informazioni su un particolare tema cerca di individuare le etichette aventi maggiore attinenza e rilevanza; la ricerca di corrispondenze nel dizionario prevede che:
	\begin{enumerate}
	\item vengano prese in esame tutte le etichette $e \in E_i$, poiché solo contemplando le chiavi primarie e i relativi sinonimi si massimizza la probabilità di ottenere riscontri positivi (maggiore copertura sintattica);
	\item vengano restituite le chiavi primarie corrispondenti alla ricerca;
	\item per ogni sinonimo $e_j \in E_i$ individuato, si restituisce la corrispondente etichetta primaria $e_0 \in E_i$.
	\end{enumerate}

	\subsubsection{Ricerca di contenuti specifici}
	La ricerca di informazioni\marginpar{Etichette e accezioni} riguardanti un tema specifico viene effettuata specificando una o più etichette, declinate nelle specifiche accezioni, che presentino agli occhi dell'utente particolare attinenza e siano dunque con maggior probabilità associate ai contenuti di interesse.

	Siano $E_s$ l'insieme delle etichette\marginpar{Insiemi di etichette} cercate e $E_c$ l'insieme delle etichette assegnate ad un generico contenuto: il primo passo consiste nel sostituire le etichette sinonimiche con le equivalenti primarie ed estendere l'insieme $E_s$ alle etichette figlie di ogni $e \in E_s$.

	A questo punto\marginpar{Corrispondenza} si possono distinguere tre casi principali, a seconda del grado di corrispondenza/attinenza dei contenuti rispetto alle etichette cercate:
	\begin{description}
	\item[Corrispondenza completa:] $E_s \subseteq E_c$ \hfill \\
	Al contenuto risultano assegnate tutte le etichette richieste dall'utente e viene quindi visualizzato in cima ai risultati di ricerca (massima attinenza).
 	\item[Corrispondenza parziale:] $E_s \cap E_c \neq \emptyset$ \hfill \\
	Al contenuto risulta assegnata parte delle etichette richieste dall'utente (media attinenza).
	\item[Nessuna corrispondenza:] $E_s \cap E_c = \emptyset$\hfill \\
	Al contenuto non risulta assegnata alcuna etichetta richiesta dall'utente (attinenza nulla).
	\end{description}

	I contenuti attinenti\marginpar{Attinenza} vengono visualizzati in ordine decrescente rispetto al numero di etichette assegnate corrispondenti a quelle richieste dall'utente:
	$$\left|{E_s \cap E_c}\right|$$

	\subsubsection{Ricerca di contenuti affini}
	La ricerca di contenuti affini consiste nell'identificare, a partire da un contenuto dato, altri la cui pertinenza rispetto al tema trattato sia massima: in questo scenario valgono le medesime considerazioni emerse nella sezione precedente, previa sostituzione di $U_e$ con l'insieme delle etichette assegnate al contenuto corrente.

	%\chapter{Progettazione}
	%\section{Entit\`a del dominio}
	%\section{Contenuti informativi}

\end{document}
